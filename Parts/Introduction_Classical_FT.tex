\section{An Introduction to Classical Field Theories}
\label{sec:Classical_FT}
In this section, we will explore classical field theories  in order to give a solid foundation to their concepts, especially locality. While field theories come in many forms like thermodynamics, electrodynamics, general relativity, the standard model of physics and even string theory, we will mainly focus on classical Lagrangian field theories.\\

In Lagrangian mechanics, as taught in undergrad physics, one uses the notion of the action functional, being a measure for the "excitednes" of a system, to find physically favoured and thus realized trajectories. A common action in this context would be
$$ S(q) = \int_{\RR^+} \left( \frac{m}{2} \dot q^i(t) \dot q^i(t) - V(q(t)) \right) dt $$
For $q\in \Ci(\RR^+, \RR^n)$ and $V\colon\Ci(\RR^+, \RR^n) \ra \RR$. Now the condition of vanishing variation $\delta S \equiv 0$ imposes two conditions
\begin{align*}
  &\Rightarrow \quad \quad \frac{m}{2} \ddot q^i(t) + \nabla^i V = 0 \quad \text{Euler-Lagrange-equations}\\
  &\Rightarrow \quad \quad \delta q^i \Big|_{t=0} = 0 \quad \text{"Boundary conditions"}
\end{align*}

Now the \textbf{boundary term} $\frac{m}{2}\delta q^i(0) \dot q^i = \alpha$ can be thought of as a 1-form on $T^*\RR^n$ such that the boundary term, using $p^i(0) = \frac{m}{2} \dot q^i$, can be writen as $\omega := \delta \alpha = \delta q^i(0) \delta p^i(0)$ which is a symplectic form.\\

This marks the starting point of a procedure called \textbf{canonical quantisation} of classical mechanics. As a Hilbert space we use $\mathcal{H} \colon= l^2(\RR^n)$. Our main goal will be to generalize this procedure.\\

\subsection{Spaces of fields and Locality}
We will usually work on a fibre bundle $\pi\colon F \lra M$ over some smooth manifold $M$. For simplicity, we will assume that $M$ is closed and without boundary, furthermore we assume that $M$ is oriented and connected.

\begin{definition}[Sections and local sections]
  A \textbf{section} of $\pi\colon F \lra M$ is a smooth map $\phi\colon  M \lra F$ such that $\pi \circ \phi = \id_M$. We call $\phi$ a \textbf{local section} if it is only defined on an open subset $U \subseteq M$ such that $\pi \circ \phi = \id_U$. We further denote the space of sections by $$\Gamma(M,F)\colon = \{ \phi \in \Ci(M,F) | \pi \circ \phi = \id_M \} \equiv \mathcal{F}$$
  We often refer to $\mathcal{F}$ as the \textbf{space of fields}. Note that if we work with a vector bundle, $\mathcal{F}$ inherits a linear structure.
\end{definition}

We are mainly interested in \textbf{locality}. Thus we work with equivalence classes of such sections that coincide in a neighbourhood of a point up to some arbitrary $k$-th derivative:

\begin{definition}
  If $p \in M$ we denote by $\Gamma(p)$ the space of local sections whose domain contains $p$.
\end{definition}

Using the thus defined local spaces allows us to define the utterly important notion of \emph{Jets of sections}:

\begin{definition}[$k$-Jets of sections]
\label{def:Jets_sections}
Let $\pi\colon F\lra M$ be a fibre bundle and $k$ any integer. We say that two local sections of $\pi$ at $p \in M$ \textbf{have the same $k$-th jet at $p$} if their partial derivatives agree at $p$ up to $k$-th order in some chart around $p$. We denote by $J^k_p F$ the set of such equivalence classes and use
$$ j^k_p \phi \colon = [(\phi, p)]_k $$
to denote such equivalence classes.
\end{definition}

\begin{rem}
As a rather abstract but interesting exercise, you can show that the above definition does not depend on the choice of coordinate charts. \emph{Hint:} Introduce multiindices $I$ such that if $U^\alpha$ is a chart for $F$, we look at
$$ \dell{^{|I|}}{x^I} (u^\alpha \circ \phi) \Big|_p $$
As we will see later, this will also introduce coordinates on the objects $J^k_p F$
\end{rem}

\begin{definition}[Jet bundles]
  Given a fibre bundle $\pi\colon  F\lra M$ and an integer $k$ we denote
  $$ J^k F \colon = \{j^k_p \phi | p \in M, \phi \in \Gamma(p) \}$$
  and $J^0 F \equiv F$ together with the maps
  \begin{align*}
    \pi_k \colon  J^k F &\lra M, \quad \textbf{k-th Jet bundles}\\
    j^k_p &\lmap p \\
    \pi^l_k\colon  J^kF &\lra J^l F, \quad 1 \leq l \leq k \\
    j^k_p  \phi &\lmap j^l_p \phi
  \end{align*}
  such that $\pi_k = \pi \circ \pi^0_k$, $\pi^l_k = \pi^l_m \circ \pi^m_k$ for $0 \leq m \leq l$. Moreover if $\phi$ is a section of our fibre bundle local in some $U\subseteq M$, we define the \textbf{Jet Prolongations}
  \begin{align*}
    j^k: \mathcal{F}_p U &\lra J^k F,\\
    (\phi, p) &\lmap j^k(\phi)(p) := j^k_p(\phi)
  \end{align*}
  such that the following diagram commutes for $i \geq k \geq l$:
  \begin{center}
  \begin{tikzcd}
    \mathcal{F}_pM \arrow[drr, swap, "j^i"] \arrow[r, "j^k"] & J^k F \arrow[rr, "\pi^l_k"] & & J^l F \arrow[r, "\pi_l"] & M\\
    & & \arrow[ul, "\pi^k_i"'] J^i F \arrow[ur, "\pi^l_i"] \arrow[urr, "\pi_i"'] & &
  \end{tikzcd}
  \end{center}
\end{definition}

The next proposition will go without its proof which can be found in (Saunders, Ch6). %TODO reference

\begin{prop}
 There exists a sequence of smooth fibre bundles
 $$ ... \lra J^kF \overset{\pi^{k-1}_k}{\lra} J^{k-1}F \lra ... \lra F \lra M $$
 for every $k$. Furthermore that maps $\pi^{k-1}_k$ are \emph{surjective} with \emph{surjective tangent map} (sumbersions).
\end{prop}

Now if $(U, \mathcal{U})$ is an adapted coordinate system for $F$ such that $(x^i, u^\alpha) \equiv u$ and thus
\begin{align*}
  \Rightarrow \quad \quad \quad U^k &= \{j^k_p \phi \colon  \phi(p) \in U\}\\
  u^k &= (x^i, u^\alpha, u_I^\alpha)\\
  u^\alpha_I(j^k_p \phi) &= \dell{^{|I|} (u^\alpha \circ \phi)}{x^I}\Big|_p
\end{align*}

Now to make precise the statement "for all $k$" in the above proposition, we need to introduce some extra structures:

\begin{definition}[Inverse/Projective systems]
\label{def:inverse_system}
  Define $Sys_x \colon = (\{x_i\}, f_{ij} | i,j \in I \subset \NN)$ with
  \begin{itemize}
    \item $\{x_i\}$ a collection of spaces (generalizes to objects in a category)
    \item $f_{ij}:x_i \lra x_j$ for all $i,j$ s.t. $i \leq j$ and
    $$ f_{ik} = f_{ij} \circ f_{jk} $$
  \end{itemize}
  We call this construct an \textbf{inverse system} or \textbf{projective system}.
\end{definition}

Now we denote by $\underset{\longleftarrow}{\lim} \ x_i$ the subset of $\Pi_i x_i$ of elements $x\in \{ x_i\}$ such that $x_i = f_{ij}(x_j) \ \ \forall j \geq i$. We call $\underset{\longleftarrow}{\lim}$ the \textbf{projective/inverse} limit of the inverse system.

\begin{definition}
  The sequence $\{J^k F\}_{n\in\NN}$ together with $\pi^k_l: J^l F \lra J^k F$ defines an inverse system (of fibre bundles). We thus use the inverse limit to define $J^\infty F := \underset{\longleftarrow}{\lim}\ J^k F$.
\end{definition}

To be precise: $J^\infty F$ is the space of equivalence classes of sections $\phi:M \lra F$ such that two sections $s_1$ and $s_2$ are equivalent, if their partial derivatives agree at all orders. We denote these eqivalence classes by $j^\infty(\phi)$. As an exercise you can show that the germs of functions surject over $J^\infty \mathcal{R}$ where $\mathcal{R} = M \times \RR \lra R$. You can also find a counterexample for the opposite statement.\\

Now the interesting question for our physical analysis will be if $J^\infty F$ can be given a smooth manifold structure. It will turn out to be very convenient to consider smooth functions on $J^\infty F$ first:

\begin{definition}
  Let $\pi: F \lra M$ be a fibre budle and let $J^kF$ denote the space of $k$-jets. Since we are dealing with regular finite dimensional manifolds, we can consider $\Ci(J^kF, P)$ for some manifold $P$. For every $l\geq k$ there are connecting maps
  $$\widetilde \pi^l_k : \Ci(J^k(E), P) \lra \Ci(J^l(E), P)$$
  which can be constructed using precomposition, thus $(\pi^k_l)^* f := \widetilde \pi^l_k (f)$. Thus we obtain a so-called \textbf{direct system}
  $$ \Ci(F) \lra \Ci(J^1F) \lra \Ci(J^2F) \lra ... $$
  with functions $f_{ij}:x_i \lra x_j$ for $i\leq j$. We further define the \textbf{direct limit}
  $$\Ci(J^\infty F) := \underset{\lra}{\lim}\ \Ci(J^k F) = \Pi_k \Ci(J^k F) / \sim$$
  where $g_i \sim g_l$ iff $\exists k \geq i, k \geq l$ s.t. $\widetilde \pi^k_i g_i = \widetilde \pi^k_l g_l$.
\end{definition}

Note that by construction $f \in \Ci(J^\infty F)$ is fully represented by functions $\widehat f_k \in \Ci(J^k F)$ which only depend on a finite number of derivatives. This will be the essence of the notion of locality which we will unfold in the following pages. Thus if we consider $f \in \Ci(J^\infty F)$ represented by $\widehat f$ one some $k$-jet then on each coordinate neighbourhood $(\pi^\infty)^{-1}(U)$ and each point $\sigma = j^\infty(\phi)(p) \in (\pi^\infty)^{-1}(U)$ we have
$$ f(\sigma) = \widehat f(x^i, u^\alpha, u^\alpha_{i_1}, u^\alpha_{i_1 i_2}, ..., u^\alpha_{i_1 ... i_k})  $$
From now on, we will, in a slight abuse of notation, not distinguish between functions on $J^\infty F$ and their representatives.


\subsection{Fréchet Manifolds}

Since field theory inherently works with infinite dimensional manifolds, we have a real need for a respective mathematical theory. We saw that if we look at such infinite dimensional manifolds in a local setting, they are "tame" in that we can consider many of their concepts using finite dimensional structures. This will bring us to the notion of a \emph{Fréchet manifold}. While one can approach this concept from a very categorial direction, we will give a less abstract introduction. Note that this part will in no way be exhaustive and not even be enough to understand everything we will use it for later on. Its goal is rather to give a feeling for the formal background which is completely omitted in many introductions to the field due to its depth and level of abstraction.

\begin{definition}[Seminorm]
\label{def:Seminorm}
  Let $V$ be a vector space. A \textbf{seminorm} is a map $|\cdot | : V \lra \RR$ that satisfies
  \begin{itemize}
    \item[1.] $|v| \geq 0 \quad \forall v\in V$
    \item[2.] $|v+w| \leq |v| + |w| \quad \forall v,w \in V$
    \item[3.] $|a\cdot v| = |a| \cdot |v| \quad \forall v\in V, a \in \RR$
  \end{itemize}
  A \textbf{family of seminorms} (for our purposes) is a set $\{|\cdot |_i\}_{i\in I}$ with $|\cdot|_i$ a seminorm for each $i \in I$.
\end{definition}

\begin{definition}[Locally convex topological vector spaces]
\label{def:Locally_convex}
  A \textbf{locally convex topological vector space} is a vector space $V$ together with a family of seminorms $\Gamma$. We denote a locally convex topological vector space with such a family of seminorms by $(V, \Gamma)$.
\end{definition}

Using families of seminorms, we can assign a unique topology to the vector space they live on:

\begin{definition}
  A family of seminorms $\Gamma$ on a vector space $V$ defines a \textbf{unique topology} $\mathcal{T}_\Gamma$ compatible with the vector space structure. The neighbourhood base of $\mathcal{T}_\Gamma$ is given by the family
  $$ B_\Gamma := \{ U^J_\epsilon | \epsilon > 0, \ J \subset I \ \text{finite} \} $$
  with $U^J_\epsilon := \{ v \in V \big| |v|_i < \epsilon \ \forall i \in J \}$
\end{definition}

Using the above notions, we arrive at a proposition that sheds further light onto such spaces.

\begin{prop}
  Let $(V,\Gamma)$ be a locally convex topological vector space. Then the following statements hold:
  \begin{enumerate}
    \item $\mathcal{T}_\Gamma$ is the finest topology in which all included seminorms are continuous.

    \item $(V, \Gamma)$ is Hausdorff iff
    $$ v = 0 \quad \Leftrightarrow \quad \left( |v|_i = 0 \quad \forall i \in I\right)$$

    \item If $(V, \Gamma)$ is Hausdorff then it is metrizable iff the family $\Gamma$ is countable (i.e. there exists a metric $d:V\times V \lra \RR^+$ s.t. the topology induced by $d$ is $\mathcal{T}_\Gamma$).

    \item Convergence of sequences is controlled by the seminorms, i.e.:
    $$ (v_n)_{n\in\NN} : v_n \lra v \quad \Leftrightarrow \quad |v_n - v|_i \lra 0 \ \ \forall i \in I $$

    \item $V$ is complete with respect to $\mathcal{T}_\Gamma$ iff every Cauchy sequence converges, i.e. iff every $(v_n)_{n\in\NN}$ with $\underset{n,m \lra \infty}{\lim} |v_n - v_m|_i = 0$ converges $\forall i \in I$.
  \end{enumerate}
\end{prop}

\begin{definition}[Fréchet spaces]
  A \textbf{Fréchet space} is a sequentially complete, Hausdorff, metrizable, locally convex vector space.
\end{definition}

We give some popular examples to illuminate the above definition:

\begin{example}~
\begin{itemize}
  \item Every Banach space is a Fréchet space.
  \item $\RR^\infty = \Pi_{n\in \NN} \RR^n$ with either the cartesian topology or the corresponding family of seminorms $\{p_n(x_1,...,x_n) = |x_1| + ... + |x_n| \}$ together with the metric $d(x,y) := \sum_i \frac{|x_i - y_i|}{2^i (1+|x_i - y_i|)}$ is a Fréchet space.
  \item The space of smooth sections on a vector bundle $V \lra M$ where $(M,g)$ is a Riemannian manifold is Fréchet with $||f||_n = \sum_{i=0}^n \sup_x |\nabla^i f(x)|$ for $n\in \NN$ with $\nabla$ a covariant derivative.
\end{itemize}
\end{example}

Our next step is to generalize the notion of a derivative (differential) to infinite dimensional spaces. This will in turn enable us to define the idea of smoothness in this infinite dimensional setting.

\begin{definition}[Gâteaux-differential]
  Let $V,W$ be locally convex topological vector spaces, $U\subset V$ open and $F:V \lra W$ then $dF(u)(v)$, the \textbf{Gâteaux-differential} of $F$ at $u\in U$ along $v\in V$, is defined as
  $$dF(u)(v) := \underset{\tau \ra 0}{\lim} \frac{F(u+\tau v) - F(u)}{\tau} = \dd{}{\tau} F(u+\tau v) \Big|_{\tau = 0}$$
  If the limit exists $\forall v \in V$, $F$ is \textbf{Gâteaux differentiable} at $u \in U$.
\end{definition}

Using this idea of a differential or derivative, we can define what it means for a homeomorphism to be smooth. Thus being able to use Fréchet spaces instead of $\RR^n$ as our target space, we can define a corresponding manifold structure:

\begin{definition}[Fréchet manifolds]
\label{def:Frechet_manifolds}
  A Hausdorff topological space $M$ is a \textbf{Fréchet manifold} if it is equipped with an atlas of homeomorphisms to open sets $U$ of a Fréchet space $V$ such that the transition functions are smooth in the sense that the Gâteaux-derivations $D^{k+1}f:U \times ... \times U \lra V$ with
  $$ Df(u)v := \underset{t \lra 0}{\lim} \frac{f(u+tv)-f(u)}{t} $$
  are continuous for all $k\in \NN$.
\end{definition}

Applying the above concept of infinite dimensional manifolds onto \ref{def:inverse_system} leads us to the following idea:

\begin{lem}
  Inverse limits of normed spaces are Fréchet.
\begin{proof}
  Consider the system $f^m_n: V_n \lra V_m$ with $(V_m, |\cdot |_n)$ a normed vector space $\forall n\in \NN$. The inverse limit of the system, denoted by $V$, is endowed with the linear maps $f^n_\infty: V  \lra V_n$. The norms $|\cdot|_n$ induce a family of seminorms on $V$ via $||\cdot ||_n := |\cdot |_n \circ f^n_\infty$, such that
  \begin{enumerate}
    \item $||v||_n = |f^n_\infty(v)|_n \geq 0 \quad \forall v \in V$

    \item $||v+w||_n = ... \leq |v|_n + |w|_n$

    \item $||av||_n = |f^n_\infty(a\cdot v)|_n = |a\cdot f^n_\infty(v)|_n = |a| \cdot ||v||_n$
  \end{enumerate}
  As can be seen in \emph{Schäfer A.2.3}, the space is also metrizable. Since
  $$ \{v = 0 \ \Leftrightarrow \ f^n_\infty(v) = 0 \ \forall n \in \NN \ \Leftrightarrow \ |f^n_\infty(v)|_n = 0 \ \forall n \in \NN \ \Leftrightarrow \ |v|_n = 0 \ \forall n \in \NN \}  $$
  which proves that our space is indeed Hausdorff which proves the lemma.
\end{proof}
\end{lem}

Now let $F\lra M$ be a fibre bundle with coordinate charts $(U_a, u_a)$. Consider the induced cover $U^\infty = \{ (\pi^0_\infty)^{-1}(U_a) \}$ and the induced maps $u^\infty_a: (\pi^0_\infty)^{-1}(U_a) \lra \RR^\infty$. This defines an atlas as can be seen in \emph{Saunders 7.2.4.}. We thus arrive at one of the marking stones of our infinite dimensional analysis:

\begin{prop}
  The infinite jet bundle $J^\infty F$ together with an atlas given by $(J^\infty F, \{u^\infty_a:(\pi^0_\infty)^{-1}(U_a))\lra \RR^\infty\}_{a\in A})$ is a Fréchet manifold.
\end{prop}

Furthermore, we can state the following:

\begin{prop}
  Let $\pi:F\lra M$ be a smooth fibre bundle. For every $k \in \NN \cup \{\infty\}$,
  \begin{itemize}
    \item $\pi^k_\infty: J^\infty F \lra J^k F$ is a smooth fibre bundle
    \item $\pi_\infty: J^\infty F \lra M$ is a smooth fibre bundle
    \item $j_\phi^\infty: M \lra J^\infty F$ is smooth $\forall \phi \in \mathcal{F}$
  \end{itemize}
\end{prop}

\subsection{The Variational Bicomplex}

\textbf{Tangent Bundle:} There are several alternative ways to progress further with differential geometry on Jet bundles:
\begin{itemize}
  \item  The tangent bundle $T_xJ^\infty F$ at a point $x\in J^\infty F$ can be seen as the limit of the system $\{ (T_{\pi^k_\infty(x)}J^kF, T\pi_k) \}$ such that the bundle $T(J^\infty F) = \bigcup_{x\in J^\infty F} T_x(J^\infty F)$ is modeled on the $T(J^k F)$ and the projection
  $$ pr_{J^\infty F}: T(J^\infty F) \lra J^\infty F $$
  is represented by $\{ pr_\infty^k = pr_{J^kF} \}$ where $pr_{J^kF}: T(J^kF) \lra J^kF$.

  \item Observe that if $\phi_t$ is a smooth 1-parameter family of sections, we can define $\dot \phi_0: M \lra TF$. But $\phi$ is a section such that $d\pi(\dot \phi_0) = \dd{}{t}(\pi(\phi_t(x))) = 0_m$ and thus $\dot \phi_0 \in \ker(d\pi)$. Hence we can associate a tangent bundle to $J^\infty F$ in the context of Fréchet manifolds via
  $$ T(J^\infty F) := \Ci(\RR, J^\infty F) / \sim $$
  where $c\sim c^\prime$ iff $c(0) = c^\prime (0)$ and further
  $$ D(\phi\circ c) (0,1) = D(\phi \circ c^\prime) (0,1) $$
  with $\phi$ any chart around $c(0) \in J^\infty F$ and $D$ denoting the Gâteaux-derivative.
\end{itemize}

These two notions can be shown to be equivalent. Now in order to unambigously deal with vector fields and their relations to derivations of the respective algebra of functions, we still need to work a bit more. The following definitions can be carried over from the finite-dimensional setting:

\begin{definition}[Complex of differential forms]
Let $F\lra M$ be a smooth fibre bundle. We define the \textbf{complex of differential forms} $\Omega^\bullet(J^\infty F)$ to be the direct limit of the sequence
$$ \Omega^\bullet(F) = \Omega^\bullet(J^0F) \lra \Omega^\bullet(J^1F)\lra ... $$
together with the morphism $d:\Omega^\bullet(J^\infty F)\lra \Omega^\bullet(J^\infty F)$ given by the collection $\{ d^k: \Omega^\bullet(J^k F) \lra \Omega^\bullet(J^k F) \}$ such that $d\circ d = 0$.
\end{definition}

We can now endow $\Omega^\bullet(J^\infty F)$ with the morphism
\begin{align*}
  \wedge: \Omega^\bullet(J^\infty F) \tens{} \Omega^\bullet(J^\infty F) &\lra \Omega^\bullet(J^\infty F)\\
  \intertext{given by the collection of morphisms}
  \{ \wedge_{k,l}: \Omega^\bullet(J^k F) \tens{} \Omega^\bullet(J^l F) &\lra \Omega^\bullet(J^{\max(k,l)} F) \}_{k,l}
\end{align*}
This in turn makes the morphism $d$ into a graded derivation of the wedge product. In the following we use the derivation to split the complex of differential forms into two parts. To this end, we first define certain derived differentials using $d$:

\begin{definition}[Horizontal differential]
  Consider a local chart on $J^kF$ given by $(x^i, u^\alpha_I)$ for $I$ running over all multiindices of length lesser than $k$. The concrete form of the maps is given by
  \begin{align*}
    x^i(j^k(\phi, p)) &:= x^i(p)\\
    u^\alpha_I (j^k(\phi, p)) &:= \dell{^{|I|} (u^\alpha \circ \phi)}{x^I} \Bigg|_p
  \end{align*}
  Both are clearly smooth and uniquely determine the equivalence class $j^k(\phi, p)$. We now define the \textbf{horizontal differential} for any $f\in\Ci(J^\infty F)$ to be
  $$ d_H f := \dell{f}{x^i} dx^i + \frac{i_1 ! ... i_m !}{k!} \dell{f}{u^\alpha_I} u^\alpha_{I,i} dx^i$$
  where $k = |I|$, $i_l$ denotes the number of occurences of $l$ in $I$, $m = \dim(M)$ and $u^\alpha_{I,i} = \partial_{x^i} u^\alpha_I$.
\end{definition}

Now for convenience, denote
\begin{align*}
  \partial^I_\alpha &:= \frac{i_1 ! ... i_m !}{k!} \dell{}{u^\alpha_I}\\
  D_i &:= \dell{}{x^i} + u^\alpha_{I,i} + \partial^I_\alpha
\end{align*}
We can thus denote $d_H f = D_i f dx^i$ and further define the \textbf{vertical differential}
$$ d_V f = (d-d_H) f = \partial^I_\alpha f (d-d_H) u^\alpha_I = \partial^I_\alpha f d_V u^\alpha_I$$
We can also prove
\begin{align*}
  &(D_j f ) (j^\infty \phi) = \dell{}{x^i}(f(j^\infty \phi))\\
  &d_H x^i = dx^i, \quad \quad d_H(u^\alpha_I) = u^\alpha_{I,i} dx^i
\end{align*}

We follow our successful splitting of the operator $d$ with a collection of useful definitions that will in turn lead us to a splitting of the complex $(\Omega^\bullet(J^\infty F), d)$:

\begin{definition}[Vertical vector fields]
Let $\pi_\infty: J^\infty F \lra M$ be the fibre bundle of infinite jets. We define the subbundle of \textbf{vertical vector fields} to be
\begin{align*}
  V(J^\infty F) &:= \ker (T\pi_\infty), \ \text{i.e. for } \chi\in T(J^\infty F)\\
  V_\chi(J^\infty F) &:= \{ x_\chi \in T_x(J^\infty F) \ \ | \ \ T\pi_k(T\pi_\infty^k x_\chi) = 0 \ \forall k \}
\end{align*}
\end{definition}

Using vertical vector fields, we can now define certain types of differential form that are classified by their behaviour when acting on vertical vector fields.

\begin{definition}~
\begin{itemize}
  \item The set of \textbf{horizontal $(p,s)$-forms} is defined as
  $$ \Omega^{(p,s)}_H := \{ \omega \in \Omega^p(J^\infty F) : \omega_\chi (x_1, ..., x_{p-s+1}, \cdot, ..., \cdot) = 0 \ \ |\ \  x_i \in V_\chi(J^\infty F) \} $$

  \item A \textbf{contact form} is a differential form $\theta$ on $J^\infty F$ that is annihilated by all jets of the form $j^\infty \phi : M \lra J^\infty F$ via pullback, i.e. $(j^\infty \phi)^* \theta = 0$

  \item The set of \textbf{vertical $(p,r)$-forms} are defined as
  $$ \Omega^{(p,r)}_V := \{ \omega \in \Omega^p(J^\infty F) : \theta_1 \wedge \theta_2 \wedge ... \wedge \theta_r \wedge \widetilde \omega \ \ |\ \  \theta_i \text{ a  contact form} \} $$
\end{itemize}
\end{definition}

For contact forms, we can attain a helpful extra property as well as a basis forming the space of contact forms:

\begin{theo}
  Contact forms generate a differential ideal denoted by $\mathcal{C}$. Its basis is given by $\theta^\alpha_I = du^\alpha_I - u^\alpha_{I,i} dx^i$.
\begin{proof}
  Let $\theta$ be a contact form, thus $(j^\infty \phi)^* \theta = 0$ for all $\phi \in \mathcal{F}$. Then for all $\alpha \in \Omega^\bullet(J^\infty F)$ we have
  \begin{align*}
    (j^\infty \phi)^* (\theta \wedge \alpha) &= (j^\infty \phi)^* \theta \wedge (j^\infty \phi)^* \alpha = 0\\
    \Rightarrow \quad (j^\infty \phi)^* d(\theta \wedge \alpha) &= d(j^\infty \phi)^* (\theta \wedge \alpha) = 0
  \end{align*}
  Turning to the basis, we see that
  \begin{align*}
    (j^\infty \phi)^* d_V u^\alpha_I &= (j^\infty \phi)^*(du^\alpha_I - u^\alpha_{I,i} dx^i)\\
    &= \dell{(u^\alpha_I \circ \phi)}{x^i} dx^i - (u^\alpha_{I,i} \circ \phi) dx^i
  \end{align*}
  Thus we have a basis of the given form, since locally any form $\theta$ splits to $\theta = d_H f + d_V f$.
\end{proof}
\end{theo}

We can now use the ideal of contact forms to investigate local sections of the bundle $\pi_\infty$:

\begin{lem}
  Let $F\lra M$ be a smooth fibre bundle. A local section $\xi$ of $\pi_\infty: J^\infty F \lra M$ is holonomic, i.e. a $\infty$-prolongation of a section of $F$, if and only if $\xi^* \mathcal{C} = 0$.
\begin{proof}
  Note that
  $$ \xi^* d_V u^\alpha_I = 0 \quad \text{iff} \quad \dell{}{x^i} (u^\alpha_I \circ \xi) = u^\alpha_{I,i} \circ \xi $$
  for all indices $I,i,\alpha$ such that $\xi$ can be constructed inductively from $u_i \circ \xi$. Thus define $\phi := \pi_\infty^0 \circ \xi$ we see that $\xi = j^\infty \phi$.
\end{proof}
\end{lem}

Using the thus defined structures, we can approach the spliting of the complex $(\Omega^\bullet(J^\infty F), d)$ by defining forms of bi-degree:

\begin{definition}[Forms of bi-degree]
  Let $F \lra M$ and $J^\infty F$ be as above. Define the space of \textbf{forms of bi-degree $(r,s)$} to be the intersection
  $$ \Omega^{r,s}(J^\infty F) := \Omega_V^{p,r}(J^\infty F) \cap \Omega_H^{s,s}(J^\infty F) $$
  where $p:= r+s$. A form $\omega \in \Omega^\bullet(J^\infty F)$ is of bi-degree $(r,s)$ iff it is of the form
  $$ \omega^{I_1 ... I_r}_{\alpha_1 ... \alpha_r ; i_1 ... i_s} \theta^{\alpha_1}_{I_1} \wedge ... \wedge \theta^{\alpha_r}_{I_r} \wedge dx^{i_1} \wedge ... \wedge dx^{i_s}$$
  where all the functions $\omega^{(\cdot)}_{(\cdot)}$ are smooth in $\Ci(J^\infty F)$.
\end{definition}

We are now in a position to "split" $(\Omega^\bullet(J^\infty F), d)$. First we define
$$ \Omega^p (J^\infty F) := \bigoplus_{r+s=p} \Omega^{r,s}(J^\infty F) $$
Since $d(dx^i)=0$ we also have $d_H(dx^i) = 0$ and $d_V(dx^i) = 0$. Now we can also use $\theta^\alpha_I = du^\alpha_I - u^\alpha_{I,i} dx^j$ such that
$$ d \theta^\alpha_I = -d u^\alpha_{I,i} \wedge dx^i = - \theta^\alpha_{I,j} \wedge dx^j $$
and thus $d_V \theta^\alpha_I = 0$ and $d_H \theta^\alpha_I = - \theta^\alpha_{I,j} \wedge dx^j$. This brings us to the following theorem, which is the defining theorem for the \emph{variational bicomplex}:

\begin{theo}[Variational bicomplex]
  The complex of differential forms on the infinite jet bundle $(\Omega^\bullet(J^\infty F), d)$ splits into the bicomplex $(\Omega^{\bullet, \bullet}(J^\infty F), d_H, d_V)$, i.e. $d^2_H = d_V^2 = 0$ and further $d_H \circ d_V = -d_V \circ d_H$. We call this complex the \textbf{variational bicomplex}.
\end{theo}


\newpage
\subsection{Local Lagrangian Field Theory}

So far, we have been working on the infinite jet bundle. Our goal is to work on $\mathcal{F}$ in a \emph{local} way. Thus we need to find some local calculus on $\mathcal{F}\times M$.\\

To this end, denote by $VF:=\ker(d\pi)$ the vertical tangent bundle. Recall that if $\phi_t$ is a smooth curve in $\mathcal{F} = \Gamma(M, F)$, $\dot\phi\big|_{t=0}$ is vertical and
$$0 = d\pi(\dot\phi\big|_{t=0}) = \dd{}{t}(\pi(\phi_t))\big|_{t=0}$$
Thus we can think of $\dot \phi_0 : M \lra VF$ as a section covering $\phi_0 = pr_F \circ \dot \phi_0$. We can express this in the following diagram:\\
\begin{center}
\begin{tikzcd}
  VF \arrow[r,"pr_\FF"] & F \arrow[dr,"\pi"] &  \\
  M \arrow[u,"\dot\phi_0"] \arrow[r,"\id"] & M \arrow[u,"\phi_0"] \arrow[r,"\id"] & M
\end{tikzcd}
\end{center}

Now we can think of sections $\Gamma(M, VF)_\phi = \Gamma(M, \phi^*VF)$. An elemet of $\Gamma(M, VF)_\phi$ is a map that associated to every $x\in M$ a vector $v_{\phi(x)} \in V_{\phi(x)}F$. We can think of $v_\phi \in \Gamma(M, \phi^*VF)$.

\begin{definition}
  We define the tangent bundles
  \begin{align*}
    T\FF &:= \Gamma(M, VF), \quad &\quad T(\FF\times M) &:= \Gamma(M, VF) \times TM,\\
    pr_\mathcal{F} &\colon T\FF \lra \FF & pr_\FF \times pr_M &\colon T(\FF\times M) \lra \FF \times M
  \end{align*}
  An element of $T_\phi\FF$ is often called an \textbf{infinitesimal variation}.
\end{definition}

Now let $E \lra M$ and $F\lra M$ be two fibre bundles with projections $\pi_E$ and $\pi_F$ respectively. Note that the following diagram commutes and thus defines the \emph{product bundle}:
$$ E\times F := \{ (e,f) \in E\times F \big| \pi_E(e)=\pi_F(f) \} $$
such that the following diagram commutes:
\begin{center}
\begin{tikzcd}
  E \times_M F \arrow[d,"pr_E"'] \arrow[r, "pr_F"] & F \arrow[d,"\pi_F"] \\
  E \arrow[r, "\pi_E"] & M
\end{tikzcd}
\end{center}

In general, this allows us to write $\Gamma(M, E \times_M F) \cong \EE \times \FF$ where we identify $\EE = \Gamma(M,E)$ and $\FF = \Gamma(M,F)$. We can also write
$$T(\EE \times \FF) \cong \Gamma(M, V(E\times_M F)) \cong \Gamma(M, VE \times_M VF) \cong T\EE \times T\FF \cong (T\EE \times \FF) \times_{\EE\times\FF} (\EE\times T\FF) $$
As an exercise, you can prove the above chain of isomorphisms. Since we can further write $T(\FF\times M) \cong (T\FF \times M) \times_{\FF\times M} (\FF \times TM)$ we can for the above statements take a pair $(\phi, x)$ and denote
$$ T_{(\phi, x)}(\FF_x M) \cong T_\phi \FF \times T_x M $$
We denote by $T_\phi \FF$ the \textbf{vertical tangent space} and by $T_x M$ the \textbf{horizontal tangent space}.\\

Taking the $\infty$-prolongation $j^\infty$ and its tangent map we obtain the following commuting diagram:
\begin{center}
\begin{tikzcd}
  T(\FF \times M) \arrow[d,"pr_\FF \times pr_M"'] \arrow[r, "Tj^\infty"] & T J^\infty F \arrow[d,"pr_{J^\infty F}"] \\
  \FF \times M \arrow[r, "j^\infty"] & J^\infty F
\end{tikzcd}
\end{center}

This leads us to the following interesting piece of information:

\begin{prop}
  The tangent map
  $$ T_{(\phi,X)}j^k \colon T_\phi \FF \times T_x M \lra T_{j^k_x \phi} J^k F $$
  is given by
  $$ (T_{(\phi, x)} j^k)(\xi_\phi, v_x) := \sum_{|I|=0}^k \dot u^\alpha_I (j^k_x \xi_\phi) \dell{}{u^\alpha_I} + v^i \left( \dell{}{x^i} + \sum_{|I|=0}^k u^\alpha_{I,i} (j^{k+1}_x \phi) \dell{}{u^\alpha_I} \right) $$
\begin{proof}
  While the proof is left as an exercise to the reader, we give the following hint: To compute tangent maps in coordinates, compute the time derivative of a path $t \lmap (\phi_t, x(t))$. For example
  $$ \dd{}{t}\left( u^\alpha_I (j^k(\phi_t, x(t))) \right)\Big|_{t=0} = \dd{}{t}\left( \dell{^{|I|}}{x^I} \phi^\alpha(t, x(t)) \right)\Bigg|_{t=0} $$
  Now set $\xi_\phi = \dot \phi_0$ and $v_x = \dot x(0)$.
\end{proof}
\end{prop}

\begin{rem}
  The above proposition induces the following maps:
  \begin{align*}
    \tau_k \colon J^k(VF) &\lra TJ^k(F), & \sigma_k \colon J^{k+1}F \times TM &\lra TJ^k F,\\
    j^k_x \dot\phi(0) &\lmap \dd{}{t}(j^k_x\phi_t)_{t=0} & (j^{k+1}_x \phi, \dot x(0)) &\lmap \dd{}{t}( j^k_{x(t)} \phi)_{t=0}
  \end{align*}
\end{rem}

Now using our results for $TJ^kF$ we obtain a splitting for $TJ^\infty F$:

\begin{theo}
  There exists a splitting of $TJ^\infty F$ that takes the form of
  \begin{center}
  \begin{tikzcd}[sep = large]
    (T\FF \times M) \times_{\FF\times M} (\FF \times TM) \arrow[r,"\cong"] \arrow[d,"j^\infty_{T\FF} \times (j^\infty_\FF \times \id)"'] & T(\FF \times M) \arrow[d, "Tj^\infty"]\\
    J^\infty(VF) \times_{J^\infty F}(J^\infty F \times_M TM) \arrow[r,"\cong"] & TJ^\infty F
  \end{tikzcd}
  \end{center}
  One can read this diagram as the splitting into
  \begin{align*}
    J^\infty(VF) & \lra TJ^\infty F \quad \textbf{vertical tangent bundle}\\
    J^\infty F \times_M TM & \lra TJ^\infty F \quad \textbf{horizontal tangent bundle}
  \end{align*}
\end{theo}


\begin{corollary}
  The vector space of vector fields on $J^\infty F$ decomposes via $\VF(J^\infty F) = \VF_{vert} \oplus \VF_{hor}$ where
  \begin{align*}
    \VF_{vert} &\cong \Gamma(J^\infty F, J^\infty(VF))\\
    \VF_{hor} &\cong \Hom(J^\infty F, TM)
  \end{align*}
  Thus we can split any $V \in \VF(J^\infty F)$ into
  $$ V = V^i \dell{}{x^i} + \sum_{|I|=0}^\infty V^\alpha_I \dell{}{u^\alpha_I} $$
  where both $V^i$ and $V^\alpha_I$ are in $\Ci(J^\infty F)$.
\end{corollary}

Now we can finally unambigously answer to the question \emph{"What is Locality?"}. Essentially, locality is the requirement that we work with $(j^\infty)^* \Omega^{\bullet, \bullet}(J^\infty F)$.

\begin{rem}
  Note that we always assume that our base manifold $M$ is orientable. There exist generalizations for non-orientable manifolds (see Deligne/Freed).
  %TODO reference
\end{rem}

\begin{definition}[Local differential forms]
  A differential form on $\FF_x M$ is called a \textbf{local differential form}, iff it is the pullback of a form on $J^\infty F$ by the $\infty$-jet prolongation $j^\infty\colon \FF_x M \lra J^\infty F$. We define the \textbf{bicomplex of local forms} as
  $$ \left( \Omega^{\bullet, \bullet}_{loc} (\FF \times M) := (j^\infty )^* \Omega^{\bullet, \bullet} (J^\infty F), d, \delta \right) $$
  such that
  $$ \delta((j^\infty)^* \alpha) := (j^\infty)^* d_V \alpha, \quad \quad d((j^\infty)^* \alpha) := (j^\infty)^* d_H \alpha $$
\end{definition}

We can also extend this \emph{local} setting to vector fields:

\begin{definition}[Local vector fields]
  A \textbf{local vector field} $\chi$ on $\FF \times M$ is a section of the tangent bundle $T(\FF\times M)$ covered by a section $\chi^\infty\colon J^\infty F \lra TJ^\infty F$ and supported on $j^\infty(\FF \times M)$. In other words, the following diagram commutes:
  \begin{center}
  \begin{tikzcd}[sep = large]
    T(\FF\times M) \arrow[d,"pr"] \arrow[r, "Tj^\infty_F"] & T J^\infty F \arrow[d,"pr"']\\
    \FF \times M \arrow[u, bend left = 60, "\chi"] \arrow[r,"j^\infty_F"]& J^\infty M \arrow[u, bend right = 60, "\chi^\infty"']
  \end{tikzcd}
  \end{center}
\end{definition}


Practically speaking, a vector field is \textbf{local}, if for any $\phi$ there exists an integer $k$ sucht that the value of $\chi_\phi \in \Gamma(M, \phi^* VF)$ at $x\in M$ depends only on the $k$-th jet of $\phi$ at $x$. We denote the set of local vector fields by $\VF_{loc}(\FF \times M)$. A local vector field in $\VF(\FF)$ is called \textbf{evolutionary}.\\

Notice that by definition, local functions $\Ci_{loc}(\FF \times M)$ are represented by a pair $(f, f^\infty)$ meaning that they factor through some infinite jet as:

\begin{center}
\begin{tikzcd}[sep=large]
  f\colon \FF \times M \arrow[d,"j^\infty"'] \arrow[r] & \RR \\
  J^\infty F \arrow[ur, "f^\infty"'] &
\end{tikzcd}
\end{center}

We also port the idea of a derivation to our newly defined setting by defining them on $\Ci_{loc}(\FF \times M)$:

\begin{definition}[Local derivations]
  A derivation $D$ of $\Ci_{loc}(\FF \times M)$ is called a \textbf{local definition}, iff there exists a vector field $\chi^\infty \colon: J^\infty F \lra TJ^\infty F$ such that
  $$ Df = (\chi^\infty f^\infty) \circ j^\infty $$
  for every $(f, f^\infty)$. We denote a local derivation by $(D, \chi^\infty)$.
\end{definition}

A proposition coming directly from usual differential geometry is the correspondence between local vector fields and local derivations:

\begin{prop}
  Local vector fields are in $1:1$ correspondence with local derivations of $\Ci_{loc}(\FF \times M)$. Moreover they form a Lie subalgebra of $\Der(C^\infty_{loc})$.
\begin{proof}
  The proof is left to the reader as it is fairly similar to the case in classical differential geometry.
\end{proof}
\end{prop}

We go on with a short remark on local coordinates: If we identify
$$ (j^\infty)^* x^i \equiv x^i, \quad \quad \quad (j^\infty)^* u^\alpha_I \equiv u^\alpha_I $$
we can interpret $dx^i \in \Omega^{0,1}_{loc}(\FF\times M)$ and $\delta u^\alpha_I \in \Omega^{1,0}_{loc}(\FF\times M)$ as a basis for $\Omega^{\bullet, \bullet}_{loc}(\FF\times M)$. This allows us to write a form $\omega \in \Omega^{p,q}_{loc}(\FF\times M)$ as
$$ \omega = \omega^{I_1 ... I_p}_{\alpha_1 ... \alpha_p i_1 ... i_q} \delta u^{\alpha_1}_{I_1}\wedge ... \wedge \delta u^{\alpha_p}_{I_p} \wedge dx^{i_1} \wedge ... \wedge dx^{i_q}  $$
for $\omega^{I_1 ... I_p}_{\alpha_1 ... \alpha_p i_1 ... i_q} \in \Ci_{loc}(\FF\times M)$, $df = D_i f dx^i$ and $\delta f = \partial^I_\alpha f \delta u^\alpha_I$. Thus if we take $X\in \VF_{loc}(\FF)$ and $\omega \in \Omega^{\bullet,\bullet}_{loc}(\FF\times M)$, we obtain a variational \emph{Cartan Calculus}
$$ \lie{X} \omega = \imath_X \circ \delta \omega + \delta \circ \imath_X \omega $$
This can be generalized to $\VF_{loc}(\FF\times M)$.


\newpage
\subsection{Classical Field Theory}

\begin{definition}[Classical Lagrangian Field Theory]
   A \textbf{Classical Lagrangian Field Theory} is a triple $(M, F, L)$ where $\pi:F\lra M$ is a smooth fibre bundle and $L \in \Omega^{0, top}_{loc}(\FF \times M)$. We call the integral
   $$ S = \int_M L $$
   the \textbf{action functional}.
\end{definition}

This means
$$ L = (j^\infty)^* L^\infty \quad \text{for} \quad L^\infty \in \Omega^{0, top}_{loc}(J^\infty F) $$
and thus
$$ L^k \left(x^i, u^\alpha, u^\alpha_{i_1}, ..., u^\alpha_{i_1 ...i_n} \right) dx^1 ... dx^n$$
If we evaluate $L$ at $\phi$, we obtain
$$ L(\phi) = L\left(x^i, \phi^i, \dell{\phi^\alpha}{x^i}, ..., \dell{^k \phi^\alpha}{x^{i_1} ... \partial x^{i_n}}\right) $$

\begin{definition}[Source forms]
  Let $\alpha \in \Omega^{1,top}_{loc}(\FF \times M)$. $\alpha$ is called a \textbf{source form}, iff it only depends on the $dx^i$ and $\delta u^\alpha$, not the $\delta u^\alpha_I$.
\end{definition}

While this definition is more of a working version and does not really explain itself, the following theorem sheds some light onto the situation:

\begin{theo}
  Let $(M,F,L)$ be a Lagrangian field theory. Then there exists a source form $EL \in \Omega^{1,top}_{loc}(\FF \times M)$ and two "boundary" forms
  $$ \alpha \in \Omega^{1,top-1}_{loc}(\FF \times M), \quad \quad \omega \in \Omega^{2,top-1}_{loc}(\FF \times M) $$
  such that
  \begin{itemize}
    \item $\delta L = EL - d \alpha$
    \item $\omega = \delta \alpha \quad \Rightarrow \quad \delta \omega = 0$
    \item $d\omega = - \delta EL$
  \end{itemize}
\end{theo}

This is the statement that by means of "integration by parts" we can write $\delta L$ as $EL = E_\alpha \delta u^\alpha \wedge dx^1 \wedge ... \wedge dx^n$ with $E_\alpha\colon J^k F \lra \RR$ which defines the \textbf{Euler-Lagrange-equations} as
$$ E_\alpha  \left(x^i, \phi^\beta, \dell{\phi^\beta}{x^i}, ..., \dell{^k \phi^\beta}{x^{i_1} ... \partial x^{i_n}}\right) $$

To prove the above statement, we would need to reformulate integration by parts in "cohomological terms", thus in terms of operators on local forms, which does not fit into the scope of this course. Doing this, allows us to define the "Euler Operator" $\mathbb{E}$ such that $\mathbb{E}(L) \equiv EL$.


\newpage
