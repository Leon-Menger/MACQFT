\section{Introduction}
\label{sec:Introduction}

The goal of Quantum Field Theory is to merge the principles of Quantum Mechanics and Classical Mechanics into one fundamental field theory. However there is currently not \textbf{one} accepted understanding of QFT but rather a huge collection of different models, approaches and techniques, partly general in nature and partly specialised to certain applications. Unlike many other physical fields, QFT has been a huge influence in the development of contemporary mathematics.\\

At its core there are two pillars supporting QFT. First of all, there are powerful calculation methods, second, there is an axiomatic understanding of QFT which is completely invariant under the specific theory at hand. In this lecture we will mainly focus on the axiomatic approach to the field.\\

Rooting in classical Lagrangian mechanics, the axiomatic frameworks (see Wightman, Segal, Haag-Kastler) propose that a QFT is an assignment that associates to $(d-1)$-manifolds Hilbert spaces (or other structures) and of $d$-manifolds to (unitary) operators. This assignment does define a functor between certain categories which sums up certain axiomatic approaches and provides a first link to more modern mathematics.\\

Most field theories which are of interest for physics (in fact most \emph{known} field theories) are so-called "gauge theories" meaning that they display a fundamental redundancy in the description of physical quantities. To treat such occurences, the methods of cohomology have been rediscovered for physical applications. This is largely due to Batalin and Vilkovisky who provided an enrichment to the axiomatic picture.\\

The main question still remains: What exactly \textbf{is} a QFT? In this lecture we will explore many ideas introduced by Wilson, Kadanoff, Polchinski and others and which have been extended and framed in recent years by Costello, Giulliani, Cattaneo, Mnev, Reshtekhin, Fredenhagen, Rejzner and others.


\newpage
