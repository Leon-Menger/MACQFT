\section{Gauge Theories}
\label{sec:Gauge_Theories}

\subsection{Quantisation in the Presence of Local Symmetries}

Recall that our chosen approach to the quantisation of (local) Lagrangian field theories deals with expressions of the form
\begin{equation}\langle O_1 ... O_n \rangle := \frac{1}{\ZF} \int_\FF e^{i/\hbar S(\phi)} O_1(\phi) ... O_n(\phi) D\phi \end{equation}
where $S = \int_M L$, the Lagrangian $L$ is a local $(0,top)$-form and $\FF = \Gamma(M, F)$. Further the $O_i$ are local action functionals on $\FF$ and we define
\begin{equation}\ZF := \int_\FF e^{i/\hbar S(\phi)} D\phi \end{equation}
where $D\phi$ is supposed to be "a measure" on $\FF$ which \textbf{does not exists in general}. Now for \textbf{finite dimensions}, given a measure $\mu$ on $X$ we have the \textbf{Stationary Phase Formula}
\begin{align}\label{eq:StatPhase}\tag{SPF}
  \ZF := \int_X \mu e^{i/\hbar S_0} \ \underset{\hbar \lra 0}{\sim} \ &\left[ \sum_{x_0 \in Crit(S_0)} e^{i/\hbar S(x_0)} \left|\det\HH[S](x_0)\right|^{-1/2} \exp\left(i \frac{\pi}{4} \sgn(S(x_0))\right) \right]\\
   &\cdot (2\pi \hbar)^{\dim(X)/2} \sum_{\gamma \ graph} \frac{\hbar^{-\chi(\gamma)}}{|\Aut(\gamma)|} \Phi_\gamma \nonumber
\end{align}
Where $\gamma$ are \emph{Feynman graphs}, $\chi(\gamma)$ is the Euler characteristic of $\gamma$ (thus $|V|-|E|$) and $\Phi(\gamma)$ is the weight of a graph. Note that here we have an \emph{implicit assumption}, namely that $\HH[S](x_0)$ is non-degenerate. In field theory, we use the stationary phase formula as a \emph{definition} of the "path integral".\\

Now we can ask, if the possible degeneracy of $\HH[S](x_0)$ really is an issue. The answer is: \textbf{Yes!} Consider $\FF = \Gamma(M, F)$ with $M$ compact without boundary. Let further $S = \int_M L$. Now assume we are given a collection $\{V_\alpha\}^M$ of local symmetries \ref{def:LocalSymmetry} such that
\begin{equation}
\label{eq:Noether}
  \lie{V_\alpha} S = V_\alpha^{(i)} \frac{\delta S}{\delta \phi^{(i)}} = 0
\end{equation}
Namely we obtain the \emph{Noether Identities}. Now consider
\begin{align}
  0 = \frac{\delta}{\delta \phi^{(j)}} \left( V_\alpha^{(i)} \frac{\delta S}{\delta \phi^{(i)}} \right) = \HH[S]_{(ij)} V^{(i)}_\alpha + \frac{\delta S}{\delta \phi^{(i)}} \frac{\delta V_\alpha^i}{\delta \phi^{(j)}}
\end{align}
Thus we restrict to $x_0 \in \Crit(S) = \{x \in \FF \ s.t. \ \delta S|_x = 0\}$ such that
\begin{equation}
  \HH[S]_{(ij)}(x_0) V^{(i)}_\alpha \overset{\nequiv}{=} 0 \quad \Leftrightarrow \quad V_\alpha \ \text{is a local symmetry}
\end{equation}
\begin{ex}
  Show that $\{V_\alpha \in \VF_{evol}(\FF)\}$ which are local symmetries of $S$ forms a Lie subalgebra of $\VF_{evol}(\FF)$. We further denote it by $\widetilde{\gf}$.
\end{ex}
Note that not all symmetries are "interesting" and thus not all symmetries are created equal. Consider a bivector field $\mu \in \Gamma(\bigwedge^2 (T\FF))$ and look at the vector field
\begin{equation}
  \omega := \mu(\delta S), \quad \mu = \mu^{ij} \frac{\delta}{\delta \phi^{(i)}} \wedge \frac{\delta}{\delta \phi^{(j)}}, \quad \omega = \mu^{ij} \frac{\delta S}{\delta \phi^{(i)}} \frac{\delta S}{\delta \phi^{(j)}}, \quad \mu^{ij} = - \mu^{ji}
\end{equation}
Now we claim that $\omega$ is a local symmetry $\forall \mu$:
\begin{proof}
  $\lie{\omega} S = \frac{\delta S}{\delta \phi^{(i)}} \omega^i = \frac{\delta S}{\delta \phi^{(i)}} \frac{\delta S}{\delta \phi^{(j)}} \mu^{ij} \equiv 0$
\end{proof}
This leads us to the following definition:

\begin{definition}[Trivial Symmetries]
  A local symmetry of the type $\omega = \mu(\delta S)$ for $\mu \in \Gamma(\bigwedge^2 (T\FF))$ is called a \textbf{trivial symmetry}. Denote the space of such trivial symmetries by $\tf$.
\end{definition}

\begin{lem}
  $\tf$ is an ideal in $\widetilde{\gf}$, thus $[\widetilde{\gf}, \tf] \subset \tf$.
\begin{proof}
  The proof is left as an exercise to the reader.
\end{proof}
\end{lem}

\begin{rem}
  Note that if $\omega$ is trivial, $\lie{\omega}S = 0$ does \textbf{not} impose nontrivial conditions, thus no Noether Identities. They are not associated to nontrivial conservation laws, currents, charges etc. Since they particularly vanish on $\Crit(S)$ they are still considered symmetries.
\end{rem}

When considering theories with symmetries, we effectively want to work with a kind of quotient
\begin{equation}
  \gf := \widetilde{\gf}/\tf
\end{equation}
Notice that $\gf$ is not necessarily a Lie subalgebra:
\begin{itemize}
  \item $[\tf, \gf] \subset \tf$ thus $\tf$ is an ideal in $\gf$.
  \item $[\gf, \gf] = \tf + \gf$ thus $\gf$ is closed "on-shell".
\end{itemize}
Thus \textbf{$\gf$ is a subalgebra at least on $\Crit(S)$}. There are two main approaches to this:

\begin{itemize}
  \item Dealing with $\gf$'s that are closed everywhere (Lie subalgebras). This treatment is due to Beach, Rovet, Stara, Tyutin \textbf{(BRST)}
  \item More generally we search for a formalism that can treat Lie algebras $\gf$ that only form Lie subalgebras on $\Crit(S)$. This leads to the Batalin-Vilkovisky formalism \textbf{(BV)}.
\end{itemize}

Most cases of physical interest are such that $\gf$ is a Lie subalgebra but the conceptual generalisation of $BV$ will allow for
\begin{itemize}
  \item more general \textbf{gauge-fixing conditions},
  \item a treatment of boundary data,
  \item observables in $BV$ that are more general than those in $BRST$.
\end{itemize}

Thus for the rest of the script assume that $\gf$ is a subalgebra.


\subsection{Lie algebra actions \& Lie algebra cohomology}

\begin{definition}[Lie group/algebra actions]
  Let $G$ be a Lie group with lie algebra $\gf$ and let $M$ be a smooth manifold.
  \begin{itemize}
    \item A  \textbf{Lie group action} is a smooth map $R\colon G \times M \lra M$ such that for all $g \in G$ the map $R_g \colon M \lra M$ is a diffeomorphism. This is equivalent to $\widetilde{R} \colon G \lra \Diff(M)$ being a group homomorphism.

    \item A \textbf{Lie algebra action} on $M$ is a Lie algebra morphism
    \begin{equation}\rho \colon \gf \lra \VF(M) \end{equation}

    \item If $\{t_a\}$ is a basis of $\gf$, the image $\{V_a = \rho(t_a)\}$ is a basis of \textbf{fundamental vector fields}.
  \end{itemize}
\end{definition}

\begin{example}
  Given $\widetilde{R}$ a Lie group action we obtain a Lie algebra action by
  \begin{equation}d \widetilde{R} |_{id} \colon \gf \lra \VF(M) \end{equation}
\end{example}

\begin{definition}[Tangent Distributions]
  Let $M$ be a smooth manifold. A \textbf{smooth (tangent) distribution} $\Delta$ of rank $k$ on $M$ is a subset $\Delta \subset TM$ such that $\forall p \in M$ the set $\Delta_p \subset T_p M$ is a $k$-dimensional subspace and such that there exists a neighbourhood $N_p$ over which there are $k$ linearly independent smooth vector fields with $\Delta_y = \Span(X_1(y), ..., X_k(y))$ for $y \in N_p$. A smooth (tangent) distribution is \textbf{involutive} if $\Gamma(\Delta)$ forms a Lie subalgebra of $\Gamma(TM)$.
\end{definition}

\begin{ex}
  Show that a Lie algebra action on $M$ induces an involutive distribution on $TM$.
\end{ex}

\begin{rem}
  Unsurprisingly, in infinite dimensions one needs to be very careful. Say that $\delta \subset TM$ is a (smooth) subbundle. In order to keep the notion of a "finite rank" $k$ one can say that $\Delta_y$ is finitely generated as a $\Ci(M)$-span. Notice that in finite dimensions \textbf{Frobenius' Theorem} states that an involutive smooth distribution is tangent to an integral submanifold $N \subset M$ s.t. $TN \simeq \Delta$. Unfortunately this does not generally hold true in infinite dimensions. It does when the manifolds are Banach, however for Fréchet manifolds it does not necessarily. \textbf{Warning:} The Inverse Function Theorem can also fail in Fréchet manifolds.
\end{rem}

Now consider a Lie algebra action $\rho \colon \gf \lra \VF(M)$ and recall that $\VF(M) \simeq \Der(\Ci(M))$.

\begin{definition}
  The space of $\Ci(M)$-valued $k$-forms on $\gf$ is defined to be
  \begin{equation}
    C^k_{CE}(\gf ; \Ci(M)) := \Hom\left(\bigwedge^k \gf, \Ci(M)\right)\cong\bigwedge^k\mathfrak{g}^*\otimes\Ci(M).
  \end{equation}
  This is called the \textbf{Chevalley--Eilenberg} complex for the Lie algebra action $\rho \colon \gf \lra \Der(\Ci(M))$. Now define the map
  \begin{equation}
    d_{CE} \colon C^k_{CE}(\gf ; \Ci(M)) \lra C^{k+1}_{CE}(\gf ; \Ci(M))
  \end{equation}
  such that the following properties hold:
  \begin{enumerate}
    \item $\forall f\in \Ci(M)$ we have $df(X) := \rho(X)(f) \ \forall X \in \gf$.
    \item $\forall \alpha \in \gf^*$ we have $d_{CE}\alpha(X,Y) := - \alpha([X,Y]) \ \forall X,Y \in \gf$.
    \item We extend to $\bigwedge^\bullet \gf^*$ using the Leibniz rule
    \begin{equation}
      d_{CE} (\alpha \wedge \beta) = d_{CE}\alpha \wedge \beta + (-1)^{|\alpha|} \wedge d_{CE} \beta
    \end{equation}
    where $|\alpha| = k$ if $\alpha \in \bigwedge^k\gf^*$.
    \item We extend to $\bigwedge^\bullet \gf^* \otimes \Ci(M)$ by
    \begin{equation}
      d_{CE}(\omega \otimes f) = d_{CE} \omega \otimes f + (-1)^{|\omega|} \omega \otimes d_{CE} f.
    \end{equation}
  \end{enumerate}
\end{definition}

\begin{prop}
  $d_{CE}^2 = 0$ and hence $(C_{CE}^\bullet(\gf ; \Ci(M))$ is a cochain complex.
\begin{proof}
  Since by property $2.$ in the definition of $d_{CE}$ it is just the dual of the Lie bracket, it satisfies the Jacobi-Identity and thus squares to zero:
  \begin{align}
    d^2_{CE} \alpha(X,Y,Z) &= - d_{CE} \left[ \alpha([X,Y],Z) + \alpha([Z,X],Y) + \alpha([Y,Z],X) \right] \\
    &= \alpha([[X,Y],Z] + [[Z,X],Y] + [[Y,Z],X]) = 0
  \end{align}
\end{proof}
\end{prop}

If $\{t_a\}$ is a basis of $\gf$ and $V_a = \rho(t_a)$, denote by $C^a \in \gf^*$ the coordinates in $\gf^*$ and structure constants $f_{ab}^c$ for the Lie algebra we obtain
\begin{equation}
  d_{CE} = C^a V_a - \frac{1}{2} f^d_{ab} C^a C^b \dell{}{C^d}
\end{equation}
Now it is immediate to prove the above proposition.\\

The cohomology of the thus arising cochain complex is
\begin{equation}
  H^k(C_{CE}^\bullet; \Ci(M)) := \frac{\ker(d_{CE}|_{C^k})}{\ker(d_{CE}|_{C^{k-1}})}
\end{equation}

\begin{rem}
  The set $H^0 (C_{CE}^\bullet; \Ci(M))$ is the set of $f \in \Ci(M)$ such that
  \begin{equation}d_{CE} f = 0 \quad \Leftrightarrow \quad d_{CE} f(X) = 0 \quad \forall X\in \gf \end{equation}
  hence $\rho(X)(f) = 0$, thus $f \in H^0 (C_{CE}^\bullet; \Ci(M))$ iff it is \textbf{invariant} under the Lie algebra action:
  \begin{equation}
    H^0 (C_{CE}^\bullet; \Ci(M)) \simeq \Ci(M)^\gf
  \end{equation}
  where $\Ci(M)^\gf$ are $\gf$-invariant smooth functions on $M$.
\end{rem}

On $\bigwedge^\bullet \gf^*$ there are the following natural operations:
\begin{align}
  \forall \alpha \in \gf^* \quad \mathcal{E}(\alpha) \colon \bigwedge^k \gf^* &\lra \bigwedge^{k+1} \gf^*, \quad \quad \omega \lmap \alpha \wedge \omega \\
  \forall X \in \gf \quad \imath(X) \colon \bigwedge^k \gf^* &\lra \bigwedge^{k-1} \gf^*, \quad \quad \omega \lmap \imath_X (\omega)
\end{align}

Now we extend $\imath_X$ by the graded Leibniz rule
\begin{equation}\imath_X(\alpha \wedge \beta) = (\imath_X \alpha) \wedge \beta + (-1)^{|\alpha|} \alpha \wedge (\imath_X \beta) \end{equation}
Now if $\{t_a\}$ and $\{\alpha^a\}$ are basis for $\gf$ and $\gf^*$ respectively, then
\begin{equation}
  d_{CE} = \mathcal{E}(\alpha^a) \rho(t_a) - \frac{1}{2} \mathcal{E}(\alpha^a) \mathcal{E}(\alpha^b) \imath_{[t_a, t_b]}
\end{equation}
where we sum over repeated indices. Now define $C^a \equiv \mathcal{E}(\alpha^a)$ and $\dell{}{C^a} \equiv \imath_{t_a}$ to obtain
\begin{equation}
  C^a C^b = \mathcal{E}(\alpha^a) \alpha^b = - \mathcal{E}(\alpha^b) \alpha^a = - C^b C^a
\end{equation}
such that
\begin{equation}
  d_{CE} = C^a V_a - \frac{1}{2} f_{ab}^d C^a C^b \dell{}{C^d}
\end{equation}
Now we need to consider the collection $\{C^a\}$ as \textbf{odd coordinates}. This corresponds to the notion of \textbf{Ghost Variables} or \textbf{Ghost Fields} in physics. Thus we can formulate
\begin{equation}
  \Ci(M \times \gf[1]) \simeq \bigwedge^\bullet \gf^* \otimes \Ci(M) \cong \Hom\left(\bigwedge^\bullet \gf, \Ci(M)\right)
\end{equation}
where $\gf[1]$ is a graded vector space, concentrated in degree $-1$. This leads us to
\begin{equation}
  C_{CE}^\bullet (\gf ; \Ci(M)) \simeq \Ci(M \times \gf[1])
\end{equation}
which allows us to think of $d_{CE}$ as a vector field of degree $1$ which squares to zero. Thus we can also think of it as a \textbf{cohomological vector field} on $M_{min}:=M \times \gf[1]$ such that
\begin{equation}
  \frac{1}{2} [d_{CE}, d_{CE}]_{\VF(M_{min})} = d_{CE}^2 = 0
\end{equation}

\subsection{BRST Formalism}

This subsection is dedicated to a concise formulation of the \textbf{Becchi--Rouet--Stora--Tyutin} (short BRST) formalism. Let $F$ be a smooth manifold endowed with a Lie algebra action
\begin{equation}
  \rho \colon \gf \lra \VF(F) \simeq \Der(\Ci(M))
\end{equation}

We associate to it the Chevally-Eilenberg complex where $F_{min} = F \times \gf[1]$
\begin{equation}
  C_{CE}^\bullet (\gf, \Ci(F)) \simeq \bigwedge^\bullet \gf^* \otimes \Ci(F) \simeq \Ci(F \times \gf[1]) = \Ci(F_{min})
\end{equation}

Now consider the product $F_{Aux} := \gf^* \times \gf^*[-1]$. We want to extend $d_{CE}$ to a cohomological vector field on
\begin{equation}
  F_{BRST} := F_{min} \times F_{Aux} = F \times \gf[1] \times \gf^* \times \gf^*[-1]
\end{equation}
Luckily on $F_{Aux}$ there is a natural cohomological vector field:

\begin{lem}
  On $F_{Aux}$ consider the "vector field" $d_{Aux} \overline{c}_a = \lambda_a$ such that $d_{Aux} \lambda_a = 0$ where we are given a basis $\{e^a\}$ of $\gf^*$, the $\overline{c}_a$ are coordinates in $\gf^*[-1]$ and the $\lambda_a$ are coordinates in $\gf^*$. Then $d^2_{Aux} = 0$ and thus it is cohomological.
\end{lem}

Note that our conventions make $\overline{c}_a$ coordinates of degree $-1$ and $\lambda$ of degree $0$.

\begin{ex}
  Show that $\left( \Ci(F_{Aux}), d_{Aux} \right) \simeq \left( \Omega^\bullet(\gf^*[-1]), d_{dR} \right)$. In particular, the cohomology of $\Ci(F_{Aux})$ vanishes.
\end{ex}

We do not want to change the cohomology of $F_{min}$ which represents invariant functions. Thus we aim for a prescription of the form
\begin{equation}F_{min} \lra F_{BRST} = F_{min} \times F_{Aux} \end{equation}
such that we only "add" fields that do not contribute to the cohomology class. This allows for the following splitting:
\begin{equation}
  \Ci(F_{BRST}) \simeq \Ci(F) \otimes \bigwedge^\bullet \gf^* \otimes S^\bullet(\gf) \otimes \bigwedge^\bullet \gf \simeq C_{CE}^\bullet (\gf, \Ci(F)) \otimes \Omega^\bullet_{dR}(\gf^*[-1])
\end{equation}
which in turn allows us to define
\begin{equation}
  d_{BRST} := d_{CE} + d_{Aux}
\end{equation}
Note that the auxilary variables (fields) are often called \textbf{Nakanishi-Lautrup fields}. Now let $S \in \Ci(F)$ such that $d_{CE} S = 0$. Namely $S$ is lifted to $F_{min}$ since it is invariant under $d_{CE}$ and thus in particular a $\gf$-invariant function. Thus use $H^0 (C_{CE}^\bullet; \Ci(M)) \simeq \Ci(M)^\gf$.\\

Now think of $S \in \Ci(F_{BRST})$ such that $S \in H^0(C^\bullet_{BRST}, d_{BRST})$. Thus applying \textbf{gauge fixing} we get for a gauge field $\psi_{gf} \in \Ci(F_{Aux})^{(-1)}$ with $|\psi_{gf}| = 1$:
\begin{equation}
  S_{gf} = S + d_{BRST} \psi_{gf} \quad \quad \Longrightarrow \quad \quad  [S_{gf}]_{BRST} = [S]_{BRST}
\end{equation}
A valid question is, why this should be relevant. To understand this, recall that if $S \in C(F)^\gf$ this means that $\HH[S](x_0)$ is degenerate. How about if we choose a different representative $S_{gf} = S + d_{BRST} \psi_{gf}$ such that $\HH[S_{gf}](x_0)$ is non-degenerate? We will pick up this thought after a short digression:

\begin{rem}[Integration on Super/Graded manifolds]
  Note that everything formulated for super manifolds also applies to $\ZZ$-graded manifolds by sending a degree to the parity. Let $E \lra M$ be a rank $m$ vector bundle over $M$ where $\dim(M) = n$. Denote by $\Pi E=: \MM$ the (super) vector bundle obtained by shifting the degree of the fibre vector spaces by $1$ (which just reverses the parity in $\ZZ/2$ (super) grading).

  \begin{definition}[Berezin line bundle]
    The (real) line bundle over $M$ defined by
    \begin{equation}Ber(\MM) := \bigwedge^{top} T^*M \otimes \bigwedge^m E \end{equation}
    is called the \textbf{Berezin line bundle} over $\MM$. A \textbf{Berezinian} is a smooth section of $Ber(\MM)$ denoted by $\mu \in \Gamma(M, Ber(\MM))$.
  \end{definition}

  \begin{definition}[Berezin Integral]
    Given a section $\mu \in \Gamma(M, Ber(\MM))$ define the integration map
    \begin{equation}\int_{\MM} \cdot \mu \colon \Ci_c(\MM) \lra \RR \end{equation}
    defined as
    \begin{equation}\int_\MM f \mu = \int_M \langle \mu, (f)_m \rangle \end{equation}
    where $(f)_m$ is the component in $\Gamma_c(M, \bigwedge^m E)$ and $\langle \cdot, \cdot \rangle \colon \bigwedge^m E \otimes \bigwedge^m E^* \lra \RR$ is the fibrewise paring.
  \end{definition}

  Now consider a vector space $V$ and its parity-shifted version $\Pi V$. Now on $\Pi V$ let there be a function given as $f = f_0 + f_1 \theta$ with $f_0, f_1 \in \RR$ and $\theta$ an \emph{odd} coordinate such that
  \begin{equation}
    \int d\theta = 0 \quad  \quad \int d\theta \theta =
  \end{equation}
  In $m$ dimensions we have
  \begin{equation}
    \int d^m \theta := \int d\theta^m \int ... \int d\theta
  \end{equation}
  such that $\int d^m \theta \ \theta^1 ... \theta^m = 1$ and zero otherwise. Now $V^0 = V_m^0 \oplus V^1_m$ such that we can decompose
  \begin{equation}
    f = f^0 + f^1_i \theta^i + f^2_{ij} \theta^i \theta^j + .
  \end{equation}
  where $f^k \in \Ci(V^0)$. Now
  \begin{equation}
    \int_V f d^n x \ d^m \theta = \int_{V^0} f^{top} \ d^n x
  \end{equation}
  Thus changing coordinates in the form $\widetilde{\theta}^i = A^i_j \theta^j$ amounts to
  \begin{equation}
    \widetilde{\theta}^1 ... \widetilde{\theta}^m = \det(A) \theta^1 ... \theta^m
  \end{equation}
  Now we enforce
  \begin{equation}
    \int d^m \widetilde{\theta} \widetilde{\theta}^1 ... \widetilde{\theta}^m = 1
  \end{equation}
  such that $d^m \widetilde{\theta} = (\det(A))^{-1} d^m \theta$. In general by pure construction
  \begin{equation}
    \langle \mu, (f)_{top} \rangle \in \Gamma\left(M, \bigwedge^n T^* M \right)
  \end{equation}
  is a top-form and can be integrated. One can generalize this entire construction by looking at
  \begin{equation}
    \widetilde{Ber}(\MM) := Ber(\MM) \otimes \bigwedge^\bullet E^* \simeq \bigwedge^n T^*M \otimes \bigwedge^m E \otimes \bigwedge^\bullet E^*
  \end{equation}
  which lets us define
  \begin{equation}
    BER(\MM) = \Gamma(\MM, \widetilde{Ber}(\MM))
  \end{equation}
  by letting $\mu \in BER(\MM)$ be non-constant along the fibres of $\rho \colon \MM \lra M$. Alternatively one can define $BER(\MM)= \Gamma(\MM, \rho^*Ber(\MM))$ as a $\Ci(\MM)$-module.

  \begin{example}
    Consider the shifted tangent bundle $\MM := \Pi TM$ and recall that $\Ci(\Pi TM) \simeq \Omega^\bullet(M)$. Now let us denote by $\tilde{f} \in \Omega^\bullet(M)$ the differential form associated to $f \in \Ci(\Pi TM)$. Define $\mu_{\Pi TM}$ as the Berezinian
    \begin{equation}\int_\MM \mu_{\Pi TM} f = \int_M \tilde{f} \end{equation}
    where we work in local coordinates $\{x^i, \theta_i\}$ and $\mu_{\Pi TM} = dx^i D\theta_i$.
  \end{example}

  Now to define the generalized version of the change of variables, consider the super vector space $\RR^{n|m} = \RC$, recall that $\RR^{n|m} = \RR^n \oplus \Pi \RR^m$ and take an endomorphism of $\RC$. Now parametrise the endomorphism, namely look at $J \in \End(\RC) \otimes \Ci(S)$ where $S$ denoted the parameter space.
  \begingroup
    \allowdisplaybreaks
    \begin{align}
    J &=
    \begin{pmatrix}
      A & B \\
      C & D
    \end{pmatrix} \\
    A &\in (\End(\RR^n) \otimes \Ci(S))_{even} \\
    D &\in (\End(\RR^m) \otimes \Ci(S))_{even} \\
    B &\in (\Hom(\RR^m, \RR^n) \otimes \Ci(S))_{odd} \\
    C &\in (\Hom(\RR^n, \RR^m) \otimes \Ci(S))_{odd}
    \end{align}
  \endgroup

  \begin{definition}[Super determinant]
    The \textbf{super determinant} of $J$ is the $\Ci(S)$-function
    \begin{equation}\sdet(J) = \det(D)^{-1} \cdot \det(A-BD^{-1} C) \end{equation}
  \end{definition}

  \begin{lem}
    The super determinant is
    \begin{itemize}
      \item Multiplicative: $\sdet(JK) = \sdet(J) \sdet(K)$
      \item $\sdet(1 + \epsilon J) = 1 + \epsilon \str(J)+ \OO(\epsilon^2)$ where $\str(J) := \tr(A) - \tr(D)$
    \end{itemize}
  \end{lem}

  \begin{corollary}
    $\sdet(\exp(J)) = \exp(\str(J))$
  \end{corollary}

  \begin{theo}
    Let $\RR^{n|m}_1$ and $\RR^{n|m}_2$ be two copies of an $(n|m)$ super space with coordinates $\{x^i, \theta_a\}_k$ for $k = 1,2$. Consider the integral w.r.t the "coordinate" Berezinian
    \begin{equation}
      \mu_1 = \prod_{i,a} dx^i d\theta_a, \quad \quad \mu_2 = \prod_{j,b} dy^j d\psi_b
    \end{equation}
    Now let $\phi \colon \RR^{n|m}_1 \lra \RR^{n|m}_2$ be a smooth map of supermanifolds and let $f \in \Ci_c(\RR^{n|m}_2)$. Then
    \begin{equation}\int_{\RR^{n|m}_2} \mu_2 f = \int_{\RR^{n|m}_2} \mu_1 \sgn(\det(J_f^{nn})) \cdot \sdet(J_f) \phi^* f \end{equation}
    where
    \begin{align}
    J_f =
    \begin{pmatrix}
      J_f^{(nn)} & J_f^{(mn)} \\
      J_f^{(nm)} & J_f^{(mm)}
    \end{pmatrix} =
    \begin{pmatrix}
      \dell{y}{x} & \dell{y}{\theta} \\
      \dell{\psi}{x} & \dell{\psi}{\theta}
    \end{pmatrix}.
    \end{align}
    Here we think of $J_f$ as a function in $\End(\RR^{n|m}) \otimes \Ci(\RR^{n|m})$.
  \end{theo}

  \begin{definition}
    Let $V \in \VF(\MM)$ for a super/graded manifold $\MM$ and let $\mu \in BER(\MM)$. The \textbf{divergence} of $V$ is the $\Ci(\MM)$ function $\diver_\mu(V) \in \Ci(\MM)$ such that
    \begin{equation}\forall f \in \Ci_c(\MM), \quad \int_\MM \mu V(f) = - \int_\MM \mu \diver_\mu(V) f\end{equation}
  \end{definition}

  \begin{lem}
  \label{lem:log}
    Let $\mu_0, \mu_1 \in BER(\MM)$ with
    \begin{equation}\mu_1 = \rho \mu_0 \quad \quad  \rho \in \Ci(\MM) \end{equation}
    then $V \in \VF(M)$, $\diver_{\mu_1}(V) = \diver_{\mu_0}(V) + V(\log(\rho))$
  \begin{proof}~
    \begin{align}
      \int \diver_{\mu_1}(V) f\mu_1 &= -\int \mu_1 V(F) = - \int \rho \mu_0 V(f)\\
      &= - \int \mu_0 V(\rho f) \mu_0 + \int V(\rho) f \mu_0 \frac{\rho}{\rho} \\
      &= \int \diver_{\mu_0} (V) (\rho f) \mu_0 + V(\log(\rho)) f \mu_0 \rho
    \end{align}
  \end{proof}
  \end{lem}

  \begin{lem}
    Let $\{x^i, \theta_a\}$ denote coordinates on the super manifold $\RR^{n|m}$ such that
    \begin{equation}
      \mu_{cord} = d^n x^i \prod_a D\theta_a
    \end{equation}
    Thus we write a vector field locally as
    \begin{equation}
      v = v^i \dell{}{x^i} + v^a \dell{}{\theta_a}
    \end{equation}
    its divergence reads
    \begin{equation}
      \diver_{\mu_{coord}} v = \partial_{x^i} v^i - (-1)^{|v|} \partial_{\theta_a} v^a
    \end{equation}
  \end{lem}
\end{rem}

\begin{lem}
  \begin{align}
    \diver_{\mu_{coord}} (d_{Aux}) &= 0 \\
    \diver_{\mu_{coord}} (d_{BRST}) &= \diver_{\mu_{coord}} (d_{CE}) = \sum_a c^a \left( \sum_b f^b_{ba} + \diver_F v_a \right)
  \end{align}
  where $v_a = \rho(t_a)$ for $\{t_a\}$ a basis of $\gf$.
\begin{proof}
  First note that
  \begin{equation}\diver_{\mu_{coord}} (d_{Aux}) = - \dell{}{\overline{c}_a} \lambda_a \equiv 0 \end{equation}
  which leads us to
  \begin{equation}\diver_{\mu_{coord}} (d_{CE}) = \dell{}{x^i} (c^a v_a^i) - \dell{}{c^d} \left(- \frac{1}{2} f^d_{ab}c^a c^b\right) = c^a \diver_F (v_a) + f^d_{db}c^b \end{equation}
\end{proof}
\end{lem}

This lemma lets us formulate and prove the main theorem of BRST:

\begin{theo}[BRST]
  Let $\rho \colon \gf \lra \VF(F)$ be a Lie algebra action on $F$ and let $\tilde{\mu}$ be a Berezinian on $F_{BRST}$ such that $\diver_{\tilde{\mu}}(d_{BRST}) = 0$. Define
  \begin{equation}\lrangle{h}_\psi := \frac{\int_{F_{BRST}} \exp(\tfrac{i}{\hbar} S_{gf})h \tilde{\mu}}{\int_{F_{BRST}} \exp(\tfrac{i}{\hbar} S_{gf})\tilde{\mu}} \end{equation}
  such that
  \begin{align}
      &1. \quad \quad S_{gf} = S + d_{BRST} \psi, \ \ \psi \in \Ci(F_{BRST})^{(-1)}\\
      &2. \quad \quad d_{BRST} S = 0, \ \ d_{BRST} h = 0
  \end{align}
  Then the following statements hold true:
  \begin{itemize}
    \item[a.] $\lrangle{h}_\psi$ is locally constant in $\psi$
    \item[b.] If $a = d_{BRST} b \ \Rightarrow \ \lrangle{a}_\psi = 0 \ \forall \psi$
  \end{itemize}
\begin{proof}
  Let $\psi_t$ be a smooth family of "gauge fixing functions". Define
  \begin{equation}
    I_{\psi_t} = \int_{F_{BRST}} \exp(\tfrac{i}{\hbar} (S+ d_{BRST} \psi_t)) h \tilde{\mu}
  \end{equation}
  such that we can denote
  \begin{align}
    \dd{}{t} I_{\psi_t} &= \frac{i}{\hbar} \int_{F_{BRST}} d_{BRST}\left(\dd{}{t}\psi_t\right) \exp(\tfrac{i}{\hbar} (S_{gf}(t))) h \tilde{\mu} \\
    &= \frac{i}{\hbar} \int_{F_{BRST}} d_{BRST} \left( \dd{}{t}\psi_t \exp(\tfrac{i}{\hbar} S_{gf}(t)) h \right) \tilde{\mu} \\
    &= - \frac{i}{\hbar} \int_{F_{BRST}} (...) \diver_{\tilde{\mu}}(d_{BRST}) \tilde{\mu} = 0
  \end{align}
  Now for the second part note that
  \begin{align}
    \int_{F_{BRST}} \exp(i/\hbar S) d_{BRST} b \ \tilde{\mu} = \int_{F_{BRST}} d_{BRST} (...) \tilde{\mu} = - \int_{F_{BRST}} \diver_{\tilde{\mu}}(d_{BRST}) \tilde{\mu} = 0
  \end{align}
  which proves our theorem.
\end{proof}
\end{theo}

Now given a Berezinian $\tilde{\mu}$ we always have $\diver_{\tilde{\mu}}(d_{BRST}) = 0$. Thus BRST integrals descend to the BRST cohomology. Namely the value of the integrals do not change if you change the representative of a cohomology class.
\begin{equation}
  \int_{F_{BRST}} \tilde{\mu} \ : H^\bullet_{BRST} (\Ci(F_{BRST})) \lra \RR
\end{equation}
If $S$ has a degenerate Hessian, we make our transition to $S_{gf} = S + d_{BRST} \psi$. However the divergence-free condition is a strong restriction. In general we are interested in $d_{BRST}$-closed quantities like $\exp(i/ \hbar S_{gf}) h$ but not in $d_{BRST}$-exact quantities.\\

When $\gf$ is unimodular then $\sum f^b_{ba} = 0$ and we can find a $\gf$-invariant measure on $F$, then $\diver(v_a) \equiv 0$ and thus in particular $\diver(d_{BRST}) = 0$.

\subsubsection{Interpretation of "gauge fixing"}
Consider, as before, a manifold $F$ together with a Lie group/algebra action of $G$ (or $\gf$). Further suppose that we are given an invariant function $S \colon F \lra \RR$ where $S \in \Ci(F)^G$ and an invariant volume form $\mu_F \in \Omega^n(F)^G$ (if $F$ is non compact, take $\mu_F$ compactly supported). The action of the Lie group traces out orbits along which $S$ is constant. Let us look at the integral
\begin{equation}
  I = \int_{F} \mu_F e^{i/\hbar S} \quad \lra \quad  I = \vol_G \int_{F / G} \widetilde{\mu}_F \ e^{i/\hbar \widetilde{S}}
\end{equation}

where $\rho \colon F \lra F/G$ projects to equivalence classes (the space of leaves of the foliation defined by the group action), and $\widetilde{S} \in \Ci(F/G)$ is such that $S = \rho^* \widetilde{S}$, while $\widetilde{\mu} \in \Omega^{n-m}(F/G)$ is such that for a basis $\{v_a\}$ where $v_a = \rho(t_a)$
\begin{equation}
  \imath_{v_m} ... \imath_{v_1} \mu = \rho^* \mu
\end{equation}
Now this point of view does not generalise very well because a quotient can be very badly behaved and finding $\widetilde{S}$ might not be easy. Alternatively we look for a "section" of the projection $\rho$, denoted by $J$, such that $\im(J)$ cuts the $G$-orbits. Practically one looks for a function
\begin{equation}
  \phi \colon F \lra \gf
\end{equation}
such that
\begin{enumerate}
  \item $0 \in \gf$ is a regular value for $\phi$
  \item $\Sigma := \phi^{-1}(0)$ intersects the $G$-orbits $N \geq 1$ times, transversally (locally, for $U \subset F$ a neighbourhood of an intersection point $T_p U \simeq \gf \oplus T_p \Sigma$ for any $p \in \Sigma$)
\end{enumerate}

\begin{sketch}[Faddeev-Popov trick]
  This trick lets us write
  \begin{align}
    I = \frac{\vol(G)}{N} \int_\Sigma \left. \imath_{v_1} ... \imath_{v_m} \mu e^{i/\hbar S} \right|_\Sigma = \frac{\vol(G)}{N} \int_F \delta^{(m)}(\phi) \imath_{v_1} ... \imath_{v_m} \mu e^{i/\hbar S}
  \end{align}
  where $\delta^{(m)}(\phi) := \delta(\phi) \cdot \bigwedge_a d\phi^a$ is a distributional $m$-form.
\end{sketch}

\begin{lem}
  $\bigwedge d\phi^a \imath_{v_1} ... \imath_{v_m} \mu = \det(FP(x)) \cdot \mu$ where
  \begin{equation}
    FP(x) = d_x \phi \circ d_{(1,x)} \rho \colon \gf \lra \gf
  \end{equation}
  In components
  \begin{equation}
    [FP(x)]^b_a = \pair{d_x\phi^b(x)}{v_a(x)} = v_a (\phi^b)|_x
  \end{equation}
\begin{proof}
  $FP(x)$ is non-degenerate iff the intersection of $\phi^{-1}(\phi(x))$ with a $G$-orbit is transversal. In fact we have $\pair{d\phi}{v}_x = 0$ iff $v \in \ker(d\phi_x)$ which is equivalent to $v$ being tangent to $\phi^{-1}(\phi(x))$. Now assume transversality, i.e. non-degeneracy, and consider $V_x = \Span\{v_a^{(x)}\} \subset T_x F$. Now if we write the orbit as
  \begin{equation}
    \OO_x := \{xg \ | \ g \in G \}
  \end{equation}
  we observe $V_x \in T_x \OO$. Moreover if $\{\alpha_1, ..., \alpha_{n-m}\}$ is a basis, we can define
  \begin{equation}
    Ann(V_x) := \{ \alpha \in T_x F | \pair{\alpha}{v} = 0 \ \forall v \in V \} \subset T^*_x F
  \end{equation}
  Now a basis for $T^*_x F$ is $\{d\phi^1(x), ..., d\phi^m(x), \alpha_1, ..., \alpha_{n-m} \}$. Thus we define
  \begin{equation}
    \mu = \bigwedge_{a=1}^m d\phi^a  \wedge \alpha_1 \wedge ... \wedge \alpha_{n-m}.
  \end{equation}
  With a little care we obtain
  \begin{align}
    \imath_{v_1} ... \imath_{v_m} \mu &= \left( \sum_{\sigma \in S_m} (-1)^S \prod_{a = 1}^m \pair{d\phi^a(x)}{v_{S(a)}(x)} \right) \alpha_1 \wedge ... \wedge \alpha_{n-m}\\
    &= \det(FP(x)) \cdot \alpha_1 \wedge ... \wedge \alpha_{n-m}.
  \end{align}
  Now conclude the proof by "wedging" with $\bigwedge d\phi^a$ to arrive at the statement of the lemma.
\end{proof}
\end{lem}

We will need to keep this lemma in mind when going forward. Another helpful lemma is the following.

\begin{lem}
  Denote by $Dc D\overline{c}$ the coordinate Berezinians on $\gf[1] \oplus \gf^*[-1]$. Then
  \begin{equation}
    \det(FP(x)) = \left(\frac{\hbar}{i}\right)^m \int_{\gf[1] \oplus \gf^*[-1]} Dc \ D\overline{c} \exp(i/\hbar \pair{\overline{c}}{FP(x) \cdot c})
  \end{equation}
  The splitting $\gf[1] \oplus \gf^*[-1]$ was the original interpretation of \emph{ghosts/antighosts}. Moreover we now have an integral representation of the delta function in the form of
  \begin{equation}
    \delta(\phi(x)) = \frac{1}{(2\pi\hbar)^m} \int_{\gf^*} d^m \lambda e^{i/\hbar \pair{\lambda}{\phi(x)}}
  \end{equation}
\end{lem}

\begin{theo}[Faddeev-Popov]
\label{theorem:FP}
  \begin{align}
    I &= \int_F \exp(\tfrac{i}{\hbar} S) = \frac{\vol(G)}{N} \int_F \mu \delta(\phi) \det(FP) \exp(\tfrac{i}{\hbar} S) \\
    &= \frac{\vol(G)}{N (2\pi i)^m} \int_{F \times \gf[1] \times \gf^*[-1] \times \gf^*} \mu \ Dc \ D\overline{c} \ d^m \lambda \ \exp\left( \frac{i}{\hbar} \left( S + \pair{\lambda}{\phi(x)} + \pair{\overline{c}}{FP(x) c} \right)\right)
  \end{align}
  We define $S_{FP} := \left( S + \pair{\lambda}{\phi(x)} + \pair{\overline{c}}{FP(x) c} \right)$.
\end{theo}

\begin{theo}[FP within BRST]
\label{theorem:FP_BRST}
  Choose $\psi = \pair{\overline{c}}{\phi} \in \Ci(F_{BRST})^{(-1)}$. Then
  \begin{equation}
    S_{gf} = S + d_{BRST} \psi = S_{FP}
  \end{equation}
\begin{proof}
  Note that due to $d_{Aux} \overline{c} = \lambda$ we get
  \begin{align}
    d_{BRST} \psi = \pair{d_{BRST} \overline{c}}{\phi} - \pair{\overline{c}}{d_{BRST} \phi} = \pair{\lambda}{\phi} - \pair{\overline{c}}{d_{BRST} \phi}
  \end{align}
  such that we can write
  \begin{align}
    d_{BRST} \phi = c^a v_a (\phi) = - \pair{d\phi}{v_a} c^a
  \end{align}
\end{proof}
\end{theo}

\begin{rem}
  FP \ref{theorem:FP} is quite restrictive if you formulate it in a mathematically precise way. Nevertheless we \emph{can} intepret it as a cohomological procedure which works with fewer assumptions. However this is not fully satisfying for the following reasons:
  \begin{enumerate}
    \item Cumbersome minimal/nonminimal extension is \textbf{needed} to introduce $\psi$.
    \item BRST is inherently limited to Lie algebra actions. Thus $\widetilde{\gf} = \tf \oplus \gf$ where $\gf$ is a subalgebra.
    \item The space of invariant functions, in physical terminology called \textbf{"The Observables"} are quantities that, in principle, can be measured. In BRST they do \textbf{not} "know" anything about $S$: It is not required in the definition of $H^\bullet(C^\bullet_{CE})$.
  \end{enumerate}

  Solving all of those problems we will look at functions that are invariant on $\Crit(S)$, not globally. To incorporate this idea from the start will lead us to the \textbf{BV formalism}.
\end{rem}


\subsection{BV Formalism}
\label{subsec:BVFormalism}

The main idea of the BV formalism is to look at the space of functions invariant on $\Crit(S)$ and not necessarily global. This less restrictive setting will eliminate many of the problems the BRST formulation inherently attains form its rather restrictive formulation.

\subsubsection{Symplectic Preliminaries}
First let us recall some background on graded symplectic geometry:\\
If $\MM$ is a graded/super manifold, we define $\Omega^\bullet(\MM)$ as $\Ci(T[1]\MM)$. So if $\{x^i, \theta_a\}$ are coordinates on $\MM$, we can introduce $\{dx^i, d\theta_a\}$ as coordinates on the shifted fibres. Generally in $\ZZ$-grading there are two relevant degrees:
\begin{itemize}
  \item The de Rham "form degree" denoted by $\deg_{dR}$
  \item The internal degree (grade) that comes from the $\ZZ$-grading denoted by
  \begin{equation}\gr(x^i) = \gr(dx^i), \quad \quad \gr(\theta_a) = \gr(d\theta_a) \end{equation}
\end{itemize}
This is useful to decide the signs in the explicit calculations by defining the \textbf{total degree}:
\begin{equation}
  \td = \deg_{dR} + \gr
\end{equation}
Now recall that a graded $k$-symplectic manifold is a $\ZZ$-graded manifold $\MM$ together with a closed nondegenerate $2$-form $\omega$ of grade $\gr(\omega) = k$ inducing a natural isomorphism
\begin{equation}
  \omega^{\musSharp{}} \colon T\MM \atob{\sim} T^* \MM[k]
\end{equation}
Now in infinite dimensions, one speaks of \emph{weak symplectic} forms, if $d \omega = 0$, such that $\omega^\#$ is \emph{injective}. For special cases we can use the following result due to Schwarz

\begin{theo}[Schwarz]
  Let $(\MM, \omega)$ be a $(-1)$-symplectic graded manifold with body manifold $M$. Then
  \begin{enumerate}
    \item In the neighbourhood of any point of $M$, there are local coordinates on $\MM$ denoted by $\{x^i, \xi_i\}$ such that $\omega = \sum_i dx^i \wedge d\xi_i$.

    \item There exists a global (non-canonical) symplectomorphism
    \begin{equation}
      \phi \colon (\MM, \omega) \lra (T^*[-1]\MM, \omega_{can}), \quad \quad \omega_{can} := \sum_i dx^i \wedge d\xi_i
    \end{equation}
  \end{enumerate}
\end{theo}

Note that the first statement of this theorem corresponds to a generalized variant of the \emph{Darboux Theorem} from usual symplectic geometry.

\begin{definition}[Lagrangian Submanifolds]
  A submanifold $\LL \subset \MM$ for $k$-symplectic $\MM$ is a \textbf{Lagrangian submanifold} in $\MM$ iff it is maximally isotropic, i.e. $\LL$ is isotropic ($\omega|_\LL = 0$) and it is not properly contained in any other isotropic submanifold.
\end{definition}

In finite dimensions one can prove that $\LL$ being Lagrangian is equivalent to it being isotropic and coisotropic. This means that $\LL$ is isotropic and its symplectic complement $\LL^\omega$ is too. Thus we also have $\dim(\LL) = \frac{1}{2} \dim(\MM)$.

\begin{definition}[Conormal Bundle]
  Let $C$ be a submanifold of $M$. We can look at $ N^* C \subset (T^*M)|_C $ as the bundle over $C$ with fibre
  \begin{equation}
    N_x^* C := \left \{ \alpha \in T_x^*M \big| \pair{\alpha}{v} = 0 \ \forall v \in T_x C \right\}
  \end{equation}
  We call $N^* C$ the \textbf{conormal bundle} of $C \subset M$.
\end{definition}

\begin{example}[Conormal Lagrangian]
  Let $C \subset M$ be a smooth submanifold. The shifted conormal bundle
  \begin{equation}
    \LL_C := N^*[-1]C \subset T^*[-1]M
  \end{equation}
  is a Lagrangian submanifold. The proof is left as an interesting exercise to the reader.
\end{example}

Another important theorem, which will be given without proof due to the lack of time in this course, is the following result from symplectic geometry:

\begin{theo}[Weinstein Tubular nbhd Theorem in $(-1)$-symplectic context]
  Let $\LL \subset \MM$ be a Lagrangian submanifold in $\MM$ where $\MM$ is $(-1)$-symplectic. Then there exist
  \begin{enumerate}
    \item A tubular neighbourhood $U \subseteq \MM$ of $\LL$, i.e. a vector bundle $p \colon E \lra \LL$ together with a smooth map $J \colon E \lra \MM$ such that for the zero section $0_E \colon x \in \LL \lmap 0 \in E_p$
    \begin{equation}
      J \circ 0_E = \imath, \quad \quad \imath \colon \LL \hookrightarrow \MM
    \end{equation}
    and there exists $V \subseteq E$ with $0_E(\LL) \subseteq V$ and $\LL \subseteq U$ such that $J|_V \atob{\sim} U$ is a diffeomorphism.

    \item A tubular neighbourhood $U_0 \subset T^*[-1] \LL$ (with projection $p_0$) of the zero section $\LL_0 \equiv \LL$.

    \item A symplectomorphism $\phi \colon U \atob{\sim} U_0$ such that $\phi(\LL) = \LL_0$ and $p_0 \circ \phi|_E = \phi|_\LL \circ p$.
  \end{enumerate}
\end{theo}

Thus in a neighbourhood of a Lagrangian submanifold, the ambient $(-1)$-symplectic manifold is locally symplectomorphic to $T^*[-1]\LL$.

\begin{example}[BRST Graph Lagrangian]
  Consider $F_{BRST}$ as we defined it in the previous section ($F_{BRST} = F \times \gf[1] \times \gf^*[-1] \times \gf^*$) and let $\psi \in \Ci(F_{BRST})^{(-1)}$. Then
  \begin{equation}
    \Gamma_\psi := \graph(d \psi) \subset T^*[-1]F_{BRST}
  \end{equation}
  is a Lagrangian submanifold. Denote by $\{\phi^a\}$ coordinates in $F_{BRST}$ and by $\{\xi_a\}$ coordinates in the fibre. Then
  \begin{equation}
    \graph(d \psi) = \left\{ (\phi^a, \xi_a) \ \Bigg| \ \xi_a = \dell{}{\phi^a} \psi \right\}
  \end{equation}
  We also define
  \begin{align}
    \omega |_{\Gamma_\psi} :&= \left. \pair{d\xi}{d\phi} \right|_{\Gamma_\psi} = \left. d \xi_a d\phi^a \right|_{\Gamma_\psi} \\
    &= d\phi^a d (\partial_a \psi)  = d\phi^a \partial_b (\partial_a \psi) d\phi^b= 0
  \end{align}
  The last equation holds due to the anticommutativity of odd-degree coordinates like $d\phi^a$ while the partial derivatives commute. As an exercise check that $\dim(\graph(d \psi)) = \frac{1}{2} \dim(T^*[-1]F_{BRST})$. For $\psi = 0$ we get $\Gamma_0$ equal the zero section, thus $\Gamma_\psi$ is a kind of \emph{deformation} of the zero section.
\end{example}

\subsubsection{Classification of Lagrangians}

Let $\MM$ be a graded $(-1)$-symplectic manifold. Then
\begin{enumerate}\label{Classfication_Lagrangian}
  \item for $\LL \subset \MM$ a Lagrangian submanifold, there exists a submanifold $C \subset M$ and a symplectomorphism $\phi \colon \MM \atob{\sim} T^*[-1]M$ such that $\phi(\LL) = \LL_C \equiv N^*[-1] C \subset T^*[-1]M$.

  \item for $\LL \subset \MM$ a Lagrangian submanifold obtained from $\LL_C = N^*[-1] C$ where $C \subset M$ as a graph of $d\psi$ for some $\psi \in \Ci(\LL_C)^{(-1)}$ we can use the tubular neighbourhood theorem to identify the neighbourhood of $\LL_C$ with $T^*[-1]\LL_C$.
\end{enumerate}

We say that $\LL \sim \LL^\prime$ homotopic as Lagrangian submanifolds, iff there exists a family of Lagrangian submanifolds $\LL_t$ smoothly varying over  such that $\LL_0 = \LL$ and $\LL_1 = \LL^\prime$ for $t \in [0,1]$.

\begin{definition}
  Let $(\MM, \omega)$ be a $(n|n)$-dimensional $(-1)$-symplectic graded manifold. A Berezinian $\mu$ on $\MM$ is called \textbf{compatible with} $\omega$, if there exists an atlas of Darboux charts $\{ x^i, \xi_i\}$ such that $\mu = d^nx D^n \xi$ in every chart. We also say that the triple $(\MM, \omega, \mu)$ is compatible.
\end{definition}

Now for $(\MM, \omega, \mu)$ compatible we introduce with
\begin{equation}
  \Delta_{BV} \colon \Ci(\MM) \lra \Ci(\MM)
\end{equation}
the "Batalin-Vilkovisky" Laplacian defined locally in any Darboux chart as the degree $+1$ second-order operator
\begin{equation}
  \Delta_{BV} = \sum_i \dell{}{x^i} \dell{}{\xi_i}
\end{equation}

\begin{lem}
  $\Delta_{BV}$ is well defined and $\Delta_{BV}^2 = 0$.
\begin{proof}
  Compatibility of $\omega$ and $\mu$ implies that the transition functions are unimodular i.e.
  \begin{equation}\sdet\left( \dell{(x, \xi)}{(y, \theta)} \right) = 1 \end{equation}
  Now the fact that the BV-Laplacian squares to $0$ can be seen by looking at
  \begin{equation}
    \Delta_{BV}^2 = \sum_{i,j} \dell{}{x^i} \dell{}{x^j} \dell{}{\xi_i} \dell{}{\xi_j}
  \end{equation}
  and then mapping $(j,i) \ra (i,j)$. Using antisymmetry we obtain the statement.
\end{proof}
\end{lem}

\begin{definition}
  Let $(\MM, \omega)$ be a $(-1)$-symplectic graded manifold with Berezinian $\mu$. Define the $\mu$-Laplacian
  \begin{equation}
    \Delta_\mu \colon \Ci(\MM) \lra \Ci(\MM), \quad \quad \Delta_\mu f := \frac{1}{2} \diver_\mu(X_f)
  \end{equation}
  where $\imath_{X_f} \omega = df$, i.e. $X_f$ is the Hamiltionian vector field of $f$.
\end{definition}

If $\{\cdot, \cdot\}$ denotes the Poisson-brackets, we get $X_f = \{f, \cdot\}$ such that $\{f,g \} = X_f (g)$. For a Darboux chart $\{x^i, \xi_i\}$ and assuming that $\mu = \rho \cdot \mu_{coord}$ we obtain the following result:

\begin{lem}
  $\Delta_\mu = \Delta_{BV} + \frac{1}{2} \{f, \log(\rho)\}$. Thus in particular $\Delta_{\mu_{coord}} = \Delta_{BV}$.
\begin{proof}
  First we need to check
  \begin{equation}
    \Delta_{BV} f = - \frac{1}{2} \diver_{\mu_{coord}}(X_f).
  \end{equation}
  The right hand side, by definition, is equal to
  \begin{equation}
    \diver_{\mu_{coord}}(X_f) = \dell{}{x^i} X_f^{(x^i)} - \dell{}{\xi_i} X_f^{(\xi_i)}.
  \end{equation}
  This lets us write
  \begin{align}
    \imath_{X_f} \omega = df \quad  &\Longleftrightarrow \quad  \imath_{X_f} dx^i d\xi_i = \dell{f}{x^i} dx^i - \dell{f}{\xi_i} d\xi_i \\
     \quad &\Longleftrightarrow \quad  X_f^{(x^i)} = - \dell{f}{\xi_i} \quad and \quad X_f^{(\xi_i)} = \dell{f}{x^i}
  \end{align}
  This allows us to reformulate the divergence of the Hamiltionian vector field as
  \begin{align}
    \diver_{\mu_{coord}}(X_f) &= - \dell{}{x^i} \dell{}{\xi_i} f - \dell{}{\xi_i} \dell{}{x^i} f \\
    &= - 2 \sum_i \dell{}{x^i} \dell{}{\xi_i} f = - 2 \Delta_{BV} f
  \end{align}
  Now using \ref{lem:log}
  \begin{equation}
    \diver_{\rho \mu_{coord}} (X) = \diver_{\mu_{coord}} (X) + X(\log(\rho))
  \end{equation}
  we can conclude
  \begin{equation}
    \Delta_\mu f = \Delta_{BV} f + \frac{1}{2} \{ f, \log(\rho) \}
  \end{equation}
  which proves the statement.
\end{proof}
\end{lem}

\begin{rem}
  Note that $\Delta_\mu$ does not necessarily square to $0$ as $\Delta_{BV}$ does. However it always will if $\mu$ is compatible with $\omega$, thus $\mu = \mu_{coord}$ and hence $\Delta_\mu = \Delta_{BV}$.
\end{rem}

\subsubsection{BV Integrals}
So far we merely introduced several notions about $(-1)$-symplectic manifolds. Having a well-defined integration will bring us closer to a physical theory. Recall that $Ber(\MM) = \bigwedge^n T^* M \otimes \bigwedge^m E$. If $\MM = T^*[-1] M$ we obtain $Ber(\MM) = (\bigwedge^n T^*M)^{\otimes 2}$. Similarly if $\NC$ is a graded manifold $Ber(T^*[-1]\NC)|_\NC \cong Ber(\NC)^{\otimes 2}$.

\begin{center}
\begin{tikzcd}[sep = tiny]
  \bigwedge^n T^* \NC \arrow[dr] & \otimes & \bigwedge^m T^*\NC \arrow[dl] \\
  & T^* M &
\end{tikzcd}
\end{center}

There is a canonical map sending a Berezinian $\mu$ on $T^*[-1]\NC$ to a Berezinian on $\NC$. We denote the induced Berezinian by $\sqrt{\mu|_\NC}$. Locally, denoting mixed-parity/degree coordinates on $\NC$ by $\{X^a\}$ and with $\{X^a, \Xi_a \}$ a Darboux chart for $T^*[-1]\NC$ we can write
\begin{equation}
  \mu = \rho \mu_{coord} = \rho(X, \Xi) \prod_a DX^a D\Xi_a
\end{equation}
which is mapped to the following Berezinian on $\NC$:
\begin{equation}
  \sqrt{\mu|_\NC} := \sqrt{\rho(X, \Xi)} DX
\end{equation}

\begin{definition}[BV Integral]
  Let $(\MM, \omega, \mu)$ be compatible. A \textbf{BV integral} is defined as
  \begin{equation}
    I_\LL = \int_{\LL \subset \MM} f \sqrt{\mu|_\LL}
  \end{equation}
  where $\LL$ is a Lagrangian submanifold of $\MM$ and $f \in \Ci(\MM)$ such that $\Delta_\mu f = 0$.
\end{definition}

Now consider $T[1]M$ and $T^*[-1]M$ with coordinates $\{x^i, \xi_i \}$ and $\{x^i, \theta^i \}$ respectively. We can think of the deRham differential on $M$ as a cohomological vector field
\begin{align}
  &D \colon \Ci(T[1]M) \simeq \Omega^\bullet(M) \lra \Omega^\bullet(M) \\
  &D^2 = 0, \quad D = \theta^i \dell{}{x^i}, \quad \theta^i = dx^i
\end{align}
Now if $\vol_M = \rho(x) dx^1 ... dx^n$ is a volume form on $M$ we arrive at the following statement:

\begin{lem}
  The following two statements hold true when we consider coordinate transformations of the form $\widetilde{\theta} = \widetilde{\theta}(x, \theta)$ or $\widetilde{\xi} = \widetilde{\xi}(x, \xi)$:
  \begin{enumerate}
    \item $ \int d^n \widetilde{\theta} \ \widetilde{\rho}^{-1} = \int d^n \theta \ \rho^{-1} $
    \item $ \int d^n \widetilde{\xi} \ \widetilde{\rho} = \int d^n \xi \ \rho $
  \end{enumerate}
\begin{proof}
  First note that
  \begin{equation}
    \widetilde{\vol}_M = \vol_M \quad \Lra \quad \widetilde{\rho} = \left[ \det \dell{\widetilde{x}}{x} \right]^{-1} \rho
  \end{equation}
  However
  \begin{align}
    d^m \widetilde{\theta} &= \left[ \det \dell{\widetilde{x}}{x} \right] d^m \theta \\
    d^m \widetilde{\xi} &= \left[ \det \dell{\widetilde{x}}{x} \right]^{-1} d^m \xi
  \end{align}
  which proves the statement.
\end{proof}
\end{lem}

Now we define an adapted version of the well-known \emph{Fourier Transformation}:

\begin{definition}[Odd Fourier Transformation]
  Let $f \in \Ci(T^*[-1]M)$ with $f = f(x, \xi)$. The \textbf{odd Fourier Transformation} is the map
  \begin{equation}
    \FF_\rho \colon \Ci(T^*[-1]M) \lra \Ci(T[1]M)
  \end{equation}
  given by its action on functions $f$ as above by
  \begin{equation}
    \widetilde{f} := \FF_\rho [f](x, \theta) := \int d^n \xi \ \rho \ e^{\theta^i \xi_i} f(x, \xi)
  \end{equation}
\end{definition}

An interesting result regarding the odd Fourier Transformation is the following:

\begin{prop}
  $\FF \circ \Delta_\mu = d \circ \FF$.
\begin{proof}
  The proof is left as an exercise to the reader.
\end{proof}
\end{prop}

Further one can show that for $g \in \Ci(T[1]M)$ one can define
\begin{equation}
  \FF^{-1}[g] = \int d^n \theta \ \rho^{-1} \ e^{\xi_i \theta^i} g(x, \theta)
\end{equation}
such that
\begin{equation}
  \FF^{-1}[dg(x, \theta)] = \Delta(\FF^{-1}[g])
\end{equation}

Now we can finally formulate the main result shouldering the BV formalism:

\begin{theo}[Batalin-Vilkovisky-Schwarz Theorem]
  Let $(\MM, \omega, \mu)$ be a compatible $(-1)$-symplectic graded manifold with Berezinian $\mu$. The following statements hold true:
  \begin{enumerate}
    \item For any $g \in \Ci(\MM)$ and a Lagrangian submanifold $\LL \subset \MM$
    \begin{equation}
      \int_\LL \Delta_\mu g \sqrt{\mu_\LL} = 0
    \end{equation}

    \item Let $\LL$ and $\LL^\prime$ be two Lagrangian submanifolds whose bodies are homologous (as cycles) in the body of $\MM$ and let $f \in \Ci(M)$ such that $\Delta_\mu f = 0$. Then
    \begin{equation}
      \int_\LL f \sqrt{\mu_\LL} = \int_{\LL^\prime} f \sqrt{\mu_{\LL^\prime}}
    \end{equation}
  \end{enumerate}
\begin{proof}
  This proof is more of a sketch. You can fill in the details with the material at hand. Assume that $\mu = \rho \mu_{coord}$ and $\LL = \LL_C = N^*[-1]C$ for $C \subset M$ a submanifold of $M$. Then the BV integral can be seen as the map
  \begin{equation}
    f \lmap \int_{\LL_C} f \sqrt{\mu_{\LL_C}} = \int_C \widetilde{f}\ \Big|_C
  \end{equation}
  where $\widetilde{f}$ is the Fourier transformed $f$ and thus its form representative. The right hand side is thus seen as the integral of a differential form on $C$. Then for the first statement we get
  \begin{equation}
    \int_{\LL_C} \Delta_\mu g \sqrt{\mu_{\LL_C}} = \int_C d\widetilde{g} = 0.
  \end{equation}
  Regarding the second we have
  \begin{equation}
    \int_{\LL^\prime_C} f \sqrt{\mu_{\LL^\prime_C}} - \int_{\LL_C} f \sqrt{\mu_{\LL_C}} = \left( \int_{C^\prime} - \int_{C} \right) \widetilde{f} = \int_D d\widetilde{f} = 0.
  \end{equation}
  We obtain the last equation by seeing that
  \begin{equation}
    d \widetilde{f} = d \FF(f) = \FF \Delta_\mu f = 0
  \end{equation}
  as required per definition. Now since $\LL \sim \LL^\prime$ are homologous as cycles, $\LL \equiv \LL_C$ and $\LL^\prime = \LL^\prime_C$ such that there necessarily exists a submanifold $D \subset M$ with $\partial D = C^\prime - C$.\\

  Now for a general Lagrangian submanifold part $1)$ of the classification of Lagrangian submanifolds (see \ref{Classfication_Lagrangian}) tells us that we can reconstruct the calculation for general Lagrangian submanifolds to that of conormal type to prove the first statement of the theorem. To prove the second, we use the second part of the classification together with the following:\\

  Let $\LL_t$ be a smooth family of Lagrangian submanifolds such that for $\epsilon > 0$
  \begin{equation}
    \LL_{t+\epsilon} = \graph(\epsilon d\psi_t + \OO(\epsilon^2))
  \end{equation}
  for a function $\psi_t \in \Ci(\LL_t)^{(-1)}$. Then for $f \in \Ci(\MM)$ such that $\Delta_\mu f = 0$ we have
  \begin{equation}
    \dd{}{t} \int_{\LL_t} f \sqrt{\mu_{\LL_t}} = \int_{\LL_t} \Delta_\mu (f \cdot \psi_t) \sqrt{\mu_{\LL_t}}
  \end{equation}
  This can be separately proven by investigating
  \begin{equation}\tag{$\diamond$} \label{eq:limeq}
    \limit{\epsilon \lra 0} \ \frac{1}{\epsilon} \left[ \int_{\LL_{t + \epsilon}} f \sqrt{\mu_{\LL_{t + \epsilon}}} - \int_{\LL_{t}} f \sqrt{\mu_{\LL_{t}}} \right]
  \end{equation}
  To this end note that
  \begin{align}
    &\limit{\epsilon \lra 0} \ \ \frac{1}{\epsilon} \int_{\LL_{t}} \left( f(x, \epsilon d\psi_t + \epsilon^2) - f(x, d\psi_t) \right) \sqrt{\mu_{\LL_{t}}} \\
    = \ &\limit{\epsilon \lra 0} \ \ \frac{1}{\epsilon} \int_{\LL_{t}} \left( f(x, \epsilon d\psi_t) - f(x, d\psi_t) + \OO(\epsilon^2) \right) \sqrt{\mu_{\LL_{t}}} \\
    = \ &\int_{\LL_{t}} f(x, d\psi_t) \sqrt{\mu_{\LL_{t}}}
  \end{align}
  Now since
  \begin{align}
    \Delta_\mu(f \cdot \psi_t) &= \Delta_\mu f \cdot \psi_t + f \cdot \Delta_\mu \psi_t + \dell{f}{\xi} \dell{\psi_t}{x}\\
    &= \dell{f}{\xi} \dell{\psi_t}{x} =: f(x, d\psi_t)
  \end{align}
  we can rewrite \eqref{eq:limeq} as

  \begin{equation}
    (\diamond) \ = \ \int_{\LL_{t}} \Delta_\mu (f \cdot \psi_t) \sqrt{\mu_{\LL_{t}}}
  \end{equation}
  Since the last equation vanishes by rule of the first statement of the proof, we only need to find a family $\LL_t$ connecting $\LL$ to $\LL_C$ but this is immeadiate.
\end{proof}
\end{theo}

The BV integral vanishes on $\Delta$-exact functions because regular integral vanish on $d$-exact forms. Further it doesn't depend on a particular Lagrangian inside a family of Lagrangians like $\LL_t$ if the function is $\Delta$-closed, the analogous property holds for $d$-closed forms.

\newpage
\subsection{Classical and Quantum BV formalism}

\begin{definition}[BV Theory]
  The following set of data is called a \textbf{classical BV theory}:
  \begin{enumerate}
    \item A $\ZZ$-graded manifold $\FF$ (the space of BV fields).
    \item A $(-1)$-symplectic structure $\omega \in \Omega^2(\FF)^{(-1)}$ (the BV $2$-form).
    \item A function $S \in \Ci(\FF)^{(0)}$ satisfying the Classical Master Equation \ref{ClassicalMasterEquation} $\{S, S\}_\omega = 0$ (the BV action or master action).
  \end{enumerate}
\end{definition}

\begin{rem}
  Often it is convenient to work with the cohomological vector field $Q := \{S, \ \cdot \ \}$ or equivalently $\imath_Q \omega = dS$. Observe that $\lie{Q} \omega = 0$, i.e. $\imath_Q \omega$ is closed, but the degree reasons that we can always find $S$ given $Q$ with $\lie{Q} \omega = 0$ such that $\imath_Q \omega = dS$.
\end{rem}

An important example which includes a huge part of interesting theories is the following:

\begin{example}[BV-BRST]
  Recall the construction of BRST data $(F_{min}, d_{CE}, S_0)$ and\\ $(F_{BRST}, d_{BRST}, S_0)$. Now
  \begin{enumerate}
    \item define $\FF := T^*[-1]F_{min} \atob{p} F_{min}$ with $\omega_{BV}$ the canonical symplectic form, the same goes for $F_{BRST}$,
    \item define $S_{BV} = p^* S_0 + \widetilde{d}_{CE}$ and recall that $\widetilde{d}_{CE}$ is the odd Fourier transformed function in $\Ci(T^*[-1]F_{min})$ corresponding to a vector field on $F_{min}$, the same goes for $d_{BRST}$,
    \item define $Q_{BV} = X_{p^*S_0} + d_{CE}^{cl}$  where $d_{CE}^{cl}$ is the cotangent lift of $d_{CE}$ to $T^*[-1]F_{min}$, again the same goes for $d_{BRST}$.
  \end{enumerate}
  Now in a Darboux chart $\{\phi^a, \phi^\dagger_a\}$ we can write
  \begin{align}
    \omega_{BV} &= \sum_a d\phi^a \wedge d\phi^\dagger_a, \\
    S_{BV} &= S_0 + \sum_a \phi_a^\dagger d^a_{CE}
  \end{align}
  where $d^a_{CE} = d_{CE}(\phi^a)$, the $\phi^a$ are fields (in $\gf^*$) and $\phi_a^\dagger$ are antifields (in $\gf^*[-1]$). Then we arrive at $Q_{BV}$ such that
  \begin{equation}
    \imath_{Q_{BV}} \omega_{BV} = dS_{BV}.
  \end{equation}
\end{example}

\begin{notation}
  For $T^*[-1]F_{min}$ the coordinates $(x^i, c^a)$ introduce coordinates on the fibre $(x^\dagger_i, c^\dagger_a)$ such that
  \begin{align*}
    |x^i| &= 0 \quad \Lra \quad |x_i^\dagger| = -1, \\
    |c^a| &= 1 \quad \Lra \quad |c_a^\dagger| = -2,
  \end{align*}
  In literature one often call the $x^i$ "physical fields" (degree $0$), the $c^a$ "ghost fields" (Chevally-Eilenberg generators), the $x_i^\dagger$ "anti-fields" (degree $-1$) and the $c_a^\dagger$ "antighosts" (degree $-2$). Note that for $V^\bullet$ a graded vector space, $(V^\bullet)^{* (-k)} = (V^\bullet)^{(k)}$.
\end{notation}

Now while we already have a rough idea of physical and ghost fields, we need further investigation on the role of anti-fields and antighosts.


\subsubsection{Constructing BV data}

The main goal of this section is to construct the data necessary for a complete BV theory. This allows for an insightful description of anti-fields and antighosts. Let $F$ be a smooth finite-dimensional manifold and let ${x^i}$ be coordinates on $F$ together with a function $S \in \Ci(F)^{(0)}$.\\

The set $\left\{\phi_i := \dell{S}{x^i}\right\}$ determines an ideal $I_S \subset \Ci(F)$ of functions vanishing on $\Crit(S) := \{ x \in F \ | \ dS(x) = 0 \}$.
Assume for now that there are no nontrivial symmetries of $S$. Now we \textbf{want} a complex $(\KK^\bullet, d_\KK)$ such that its cohomology in degree $0$ is the quotient $\Ci(F)/I_S \simeq \Ci(\Crit(S))$.\\

The following is due to Koszul: Let $V^\bullet$ be a graded vector space such that $\dim(V) = \dim(F)$, e.g. $V \simeq T^*_x F$. For every coordinate $x^i$ of $V$ generate a new coordinate $x^\dagger_i \in \Ci(V[-1]) \simeq \bigwedge^\bullet V^*$. Then we can define an operator $d_\KK$ on $\Ci(F) \otimes \Ci(V[-1]) =: \KK^\bullet$ by setting
\begin{equation}
  d_\KK x^\dagger_i := \phi_i \equiv \dell{S}{x^i}, \quad \quad d_\KK x^i := 0
\end{equation}
Thus in particular $d_\KK^2 = 0$. Further for $f \in \KK^{(0)}$ we have $d_\KK f = 0 \ \Leftrightarrow \ f \in \Ci(F)$ but $f = d_\KK g \ \Leftrightarrow \ f \in I_S$ i.e. $g = \sum_i g^i x^\dagger_i$. Thus $H^0(\KK^\bullet) = \Ci(F)/I_S$.

\begin{prop}
  The \textbf{Koszul complex} $(\KK^\bullet, d_\KK)$ is a \textbf{resolution} of $\Ci(F)/I_S$.
\begin{proof}
  We want a complex $C^\bullet$ such that $H^\bullet(C^\bullet) \simeq \Ci(F)/I_S \simeq \Ci(\Crit(S))$ and further $H^{-i} (C^\bullet) = 0$ for all $i > 0$. In our case $H^{-1} \neq 0$ iff $\exists R^i_\alpha$ such that $d_\KK (R^i_\alpha x^\dagger_i) = R^i_\alpha \phi^i = 0$ since there is nothing in degree $-2$ that can compensate the cocycle $R^i_\alpha x^\dagger_i$.\\

  However $R^i_\alpha \phi^i \equiv R^i_\alpha \dell{S}{x^i} = 0$ (\textbf{Noether identity} \ref{eq:Noether}) is equivalent to there being a vector field $V_\alpha \in \VF(F)$ with $V_\alpha = R^i_\alpha \dell{}{x^i}$ and such that $V_\alpha(S) = 0$. Namely such a cocycle exists iff $S$ admits nontrivial symmetries which we ruled out in the definition. This proves the claim.
\end{proof}
\end{prop}

Now the coordinates $x^\dagger_i$ that were introduced in the BV construction are in fact \emph{Koszul generators}. When there are no nontrivial symmetries, we take
\begin{align}
  \FF = T^*[-1]F \quad \Lra \quad d_\KK = \{ S_{cl}, \cdot \} \equiv Q
\end{align}
where $Q = \sum_i \phi_i \dell{}{x^\dagger_i}$. Now in the case where nontrivial symmetries exist, we have nontrivial cocycles $R^i_\alpha x^\dagger_i \in H^{-1}(\KK^\bullet)$. Thus we introduce new variables in degree $-2$.\\

Let $d_1$ be the number of nontrivial Noether identities/symmetries. Namely it is the dimension of $\gf \subset \widetilde{\gf} = \tf \oplus \gf$. Thus we introduce $c^\dagger_\alpha \in \Ci(\RR^{d_1}[-2])$ and set $d_\KK c^\dagger_\alpha := R^i_\alpha x^\dagger_i$ such that by pure construction $d_\KK^2 = 0$. Thus we define a new complex
\begin{align}
  \KT^{(1)} :&= \Ci(F) \otimes \Ci(V[-1]) \otimes \Ci(\RR^{d_1}[-2]) \\
  &\simeq \Ci(T^*[-1]F \times \RR^{d_1}[-2])
\end{align}
Now if the Noether identities are all independent, i.e. if the symmetries are not reducible which means that $\rho \colon \gf \lra \VF(F)$ is injective, we have
\begin{equation}
  H^0(\KT^{(1)}) \simeq \Ci(F)/I_S, \quad \quad H^{-i}(\KT^{(1)}) = 0 \ \ \forall i > 0
\end{equation}
Now it could happen that there exist functions $r^\alpha_\beta \in \Ci(F)$ such that $r^\alpha_\beta R^i_\alpha = 0 \ \forall i$, namely relations among the symmetries. Thus there would exist a nontrivial cocycle $r^\alpha_\beta c^\dagger_\alpha \in H^{(-2)}(\KT^{(1)})$. Now we could repeat the procedure, adding $c^{(2)\dagger}_\beta \in \Ci(\RR^{d_2}[-3])$ and setting $d_2$ as the number of nontrivial relations such that $r^\alpha_\beta R^i_\alpha = 0$. This procedure can be inductively repeated leading us to
\begin{align}
  \KT^{(\infty)} :&= \Ci(T^*[-1]F) \otimes \Ci\left(\bigotimes_{k = 1}^\infty \RR^{d_k}[-k-1] \right)\\
  H^\bullet(\KT^{(\infty)}) = H^0(\KT^{(\infty)}) &\simeq \Ci(F)/I_S, \quad \quad H^{-i}(\KT^{(\infty)}) = 0 \ \ \forall i > 0
\end{align}
This is called the \textbf{Koszul-Tate resolution} of $\Ci(F)/I_S$ and $d_\KK$ is seen as the cohomological vector field on $T^*[-1]F \times \bigtimes_{k = 1}^\infty \RR^{d_k}[-k-1]$.\\

The following is due to Batalin and Vilkovisky: We can adjoin coordinates $c^{(k)\alpha}$ for every $c^{(k)\dagger}_\alpha$ that we have from the $\KT$ construction where $-|c^{(k)\alpha}| = |c^{(k)\dagger}_\alpha| + 1$. In other words one looks at
\begin{equation}
  BV^\bullet \Ci\left(T^*[-1]\left(F \times \bigtimes_{k = 1}^\infty \RR^{d_k}[-k-1]\right)\right)
\end{equation}
Now there exists a canonical symplectic form $\Omega_{BV}$.


\subsubsection{Construction of the BV Action}
Our next step is to reconstruct the "BV package". We start by looking at
\begin{equation}
  d_\KK x^\dagger_i = \{ S_{cl}, \cdot \} \equiv Q_0
\end{equation}
where we set $S_{cl} = S_0$. Now extend $S_0 \lmap S_1 := S_0 + x^\dagger_i R^i_\alpha c^\alpha$ and set $Q_1 := \{ S_1, \cdot \}$ such that $Q_1 c^\dagger_\alpha = d_\KK c^\dagger_\alpha \equiv R^i_\alpha x^\dagger_i$. Now we can calculate

\begin{align}
  Q_1 x^i &= \{ S_1, x^i \} = \frac{\delta}{\delta x^\dagger_k} ( S_0 + x^\dagger_j R^j_\alpha c^\alpha) \frac{\delta}{\delta x^k} x^i \\
  &= R^i_\alpha c^\alpha = c^\alpha V_\alpha(x^i) \ (= c^\alpha \rho(t_a) (x^i))
\end{align}

The last equation in brackets only holds if there is a Lie algebra action, e.g. in the BV-BRST formalism.

\begin{ex}
  Show that $\{S_0, S_1\} = 0$ iff $\{R^i_\alpha\}$ represents a local symmetry. Also show that $\{S_1, S_1\} = 0$.
\end{ex}

In order to really reproduce the Chevalley--Eilenberg cohomology, we need to further extend by

\begin{equation}
  S_2 = S_1 + \frac{1}{2} c^{(1)\dagger}_\alpha + f^\alpha_{\beta\gamma} c^\beta c^\gamma
\end{equation}

such that

\begin{equation}
  [V_\beta, V_\alpha] = f^\gamma_{\beta\alpha} V_\gamma, \quad \quad f^\gamma_{\beta\alpha} = f^\gamma_{\beta\alpha} (x) \in \Ci(F)
\end{equation}

No generally both $R^i_\alpha$ and the $f^\alpha_{\beta\gamma}$ are functions on $F$. So it is \textbf{not} guaranteed that $\{S_2, S_2\} = 0$ and thus $Q_2 := \{S_2, \cdot\}$ is not necessarily a differential. The following theorem provides the necessary information to fix such problems:

\begin{theo}(Batalin-Vilkovisky)
  $S_2$ can be extended by terms of higher degree in $c^{(k) \dagger}_\alpha$ and $c^{(k) \alpha}$ to a function $S_\infty \in BV^\bullet$ such that $\{S_\infty, S_\infty\} = 0$ satisfies the \emph{Classical Master Equation} \eqref{ClassicalMasterEquation}, i.e. $Q_\infty := \{S_\infty, \cdot \}$ is a differential and $\left(T^*[-1]\left(F \times \bigtimes_{k = 1}^\infty \RR^{d_k}[-k-1]\right), \Omega_{BV}, S_\infty\right)$ is a classical BV theory.
\begin{proof}
  This proof is a sketch but captures the main points. The result stems from cohomological perturbation theory where one observes that if for $S_n$, $\{S_n, S_n\} = 0$ one also has $d_\KK \{S_n, S_n\} = 0$. Then since the Koszul complex has vanishing cohomology, there exists $f_{n+1}$ such that $\{S_n, S_n\} = d_\KK f_{n+1}$.
\end{proof}
\end{theo}

\begin{ex}
  Show that if $\{V_\alpha\}$ comes from an injective Lie algebra action, then one obtains $(T^*[-1](F \times \gf[-1]))$ with the BV-BRST data.
\end{ex}

Now the BV complex as constructed above achieves two main things:
\begin{enumerate}
  \item It localises to $\Crit(S)$ by means of the Koszul-Tate construction.
  \item It looks at invariants over $\Crit(S)^{\gf}$. More generally than a Lie algebra, $\gf$ is a subset of symmetries which will be involutive on $\Crit(S)$.
\end{enumerate}

Consider a complement $\gf$ of trivial symmetries $\tf$ inside all symmetries $\widetilde{\gf}$ (for a function $S_{cl} \equiv S_0 \colon F \lra \RR$). We know that $\gf$ is a subalgebra "on shell", i.e. on $\Crit(S)$. Thus we construct $\KT^{(\infty)}$ such that $H^0(\KT^{(\infty)}) = H^\bullet(\KT^{(\infty)}) \simeq \Ci(\Crit(S))$
and use it as a $\gf$-module which means looking at the Chevally-Eilenberg complex for the $\gf$ action on $\Ci(\Crit(S))$. This leads us to the following result:

\begin{theo}
  $(BV^\bullet, Q_\infty) \simeq C^\bullet_{CE}(\gf, H^\bullet(\KT^{(\infty)})) = C^\bullet_{CE}(\gf, \Ci(\Crit(S)))$.
\begin{proof}
  The proof is ommited since it is very technical and goes beyond the timescope of this course.
\end{proof}
\end{theo}

\begin{corollary}
  $H^\bullet(BV^\bullet) \simeq \Ci(\Crit(S))^\gf \simeq \Ci(\Crit(S)/\gf)$.
\end{corollary}

Now this is all the data we need for a physical theory. We have a critical locust, symmetries and a set of evolution equations. Summing up, our procedure goes as follows:\\

We have $F, S_0$ and a set of nontrivial symmetries $\gf$. Now if $\gf$ is a Lie algebra, we can look at pure BRST or more generally at $(T^*[-1]F_{BRST}, Q_{BV})$. If it isn't, we \emph{need} to work with $(T^*[-1]F_{BRST}, Q_{BV})$. From here on we will take the tuple $(\FF_{BV}, \Omega_{BV}, S_{BV}, Q_{BV})$ (where technically $Q$ is induced) and construct a classical BV QFT. Note that in general, one could need an infinite number of ghosts, "ghosts for ghosts", anti-fields and antighosts. In standard situations where $\gf$ is a lie subalgebra, the data of the BV construction goes under the name of $L_\infty$-algebra (we won't have time to discuss this). From now on, we can consider BV data as $(\FF \simeq T^*[-1]M, \Omega, S, Q)$ where $\imath_Q \Omega = dS$ such that $(\Ci(\FF), Q) = BV^\bullet$ is the BV complex.

\subsection{Quantum Master Equations and Can. Transformations}

We again consider the $(-1)$-symplectic manifold $(\FF, \Omega)$ and formal power series $\Ci(\FF)\llbracket \hbar \rrbracket$ therein.

\begin{definition}
  An element $S \in \Ci(\FF)\llbracket \hbar \rrbracket$ is said to satisfy the \textbf{Quantum Master Equation} iff
  \begin{equation} \tag{$QME$} \label{QME}
    \frac{1}{2} \{S,S\} - i \hbar \Delta_\mu S = 0
  \end{equation}
  where $\mu$ is a Berezinian on $\FF$.
\end{definition}

Formally the above requirement is equivalent to $\Delta_\mu e^{i/\hbar S} = 0$.

\begin{ex}
  Prove this fact. Also explain why it is "formally" equivalent.
\end{ex}

This follows from the following fact about BV data: The BV Laplacian is \textbf{not} a derivation of the product, i.e.

\begin{align}
  &\Delta_\mu (f \cdot g) = \Delta_\mu f \cdot g + (-1)^{|f|} f \cdot \Delta_\mu g + (-1)^{|f|} \{f,g\} \\
  \Lra \quad &\Delta_\mu e^{i/\hbar S} = (-i \hbar)^{-2} \left( \frac{1}{2} \{S,S\} - i \hbar \Delta_\mu S \right) e^{i/\hbar S}.
\end{align}

Note that we say that $(\Ci(\FF), \Delta_\mu, \{\cdot, \cdot\})$ is a BV algebra if $\Delta_\mu$ is a derivation of the bracket. Sometimes it is convenient to think of power series in $(-i \hbar)$ instead of just $\hbar$. Thus

\begin{equation}
  S = S^{(0)} + (-i \hbar) S^{(1)} + ...
\end{equation}

and \eqref{QME} is equivalent to the following set of equations

\begin{align}
  &\{S^{(0)}, S^{(0)}\} = 0 \quad \quad (CME)\\
  &\{S^{(0)}, S^{(1)}\} + \Delta_\mu S^{(0)} = 0 \\
  &\{S^{(0)}, S^{(2)}\} + \frac{1}{2} \{S^{(1)}, S^{(1)}\} + \Delta_\mu S^{(1)} = 0 \\
  &... \nonumber
\end{align}

Now starting from a solution of the Classical Master Equation \eqref{ClassicalMasterEquation} for classical BV data, we can extend it to a power series in $\hbar$ that satisfies \eqref{QME} only if
\begin{equation}
  \Delta_\mu S^{(0)} = - \{S^{(0)},S^{(1)}\}.
\end{equation}
Namely the class of $\Delta_\mu S^{(0)}$ in the cohomology defined by $Q := \{S^{(0)}, \cdot\}$ vanishes (i.e. $\Delta_\mu S^{(0)} = Q$-exact). Note that $Q$ acts as a differential. This means that we can find an appropriate $S^{(1)}$ such that
\begin{equation}
  \frac{1}{2} \{S^{(1)}, S^{(1)}\} + \Delta_\mu S^{(1)} = - \{S^{(0)}, S^{(2)}\}.
\end{equation}
Now the question is if there is a $Q$-exact $S^{(1)}$ such that the above equation holds. There are indeed obstructions to this but in many cases this has a chance to work.

\begin{definition}
  If $S$ and $S^\prime$ are two solutions of the \eqref{QME} they are said to be equivalent if there exists a \textbf{BV canonical transformation}, i.e. a family $S_t \in \Ci(\FF)^{(0)}\llbracket \hbar \rrbracket, R_t \in \Ci(\FF)^{(-1)}\llbracket \hbar \rrbracket$ such that $S_0 = S, S_1 = S^\prime$ and
  \begin{equation}\tag{$\blacksquare$}\label{BVCanonical}
    \dd{}{t} S_t = \{S_t, R_t\} - i \hbar \Delta_\mu R_t
  \end{equation}
  We call $R_t$ the \textbf{generators of the canonical BV transformation}.
\end{definition}

If we denote $\{S_t, \cdot\} - i \hbar \Delta_\mu =: \delta_t$, a Quantum BV operator for $S_t$, then we can express \eqref{BVCanonical} as
\begin{equation}
  \dot S_t = \delta_t R_t
\end{equation}

\begin{rem}
  The Koszul-Tate condition eliminates cohomology in negative degrees. The problem of extending $CME \lra QME$ however is related to cohomology in positive degrees. Now in field theory, namely when working with $\infty$-dimensional manifolds, the requirement that $BV^\bullet$ has vanishing negative cohomology is a quite strong one which can hinder the above transition. A strictly better requirement is that $\dim(H^{-i}) < + \infty$.
\end{rem}

Let us look at \eqref{QME} for the family $S_t$:
\begin{equation}\tag{$QME_t$}\label{QMEt}
  \frac{1}{2} \{S_t,S_t\} - i \hbar \Delta_\mu S_t = 0.
\end{equation}
By differentiating the QME by time we thus obtain:
\begin{equation}
  \delta_t (\dot S_t) = 0.
\end{equation}

\begin{lem}
  The Quantum BV operator
  \begin{equation}
    \delta_S := \{S, \cdot\} - i \hbar \Delta_\mu = -i \hbar e^{-i/\hbar S} \Delta_\mu \left( e^{-i/\hbar S} \ \cdot \right)
  \end{equation}
  squares to zero iff $S$ satisfies the \eqref{QME}
\end{lem}

Thus we see from \eqref{QMEt} that $\delta_t^2 = 0$ which means that $\delta_t$ is a differential and further that $\dot S_t$ is $\delta_t$-closed. Additionally $\dot S_t$ is $\delta_t$-exact due to \eqref{BVCanonical} which implies that
\begin{equation}
  \dd{}{t}[\eqref{QMEt}] = 0
\end{equation}
if \eqref{QMEt} is satisfied. Hence it is satisfied for all $t$ if $S_t, R_t$ parametrise a canonical transformation. Altogether, \eqref{QMEt} is satisfied along equivalence classes of solutions.

\begin{prop}
  $\dd{}{t} e^{i/\hbar S_t} = \Delta_\mu \left( - i \hbar e^{i/\hbar S_t} R_t \right)$ and for $S \sim S^\prime$ we have $e^{i/\hbar S} - e^{i/\hbar S^\prime} = \Delta_\mu \left( - i \hbar \int_0^t dt e^{i/\hbar S_t} R_t \right)$.
\end{prop}

\begin{definition}
  Define $\sigma := S_t + dt R_t$ such that $\sigma \in \Omega^\circ([0,1]) \otimes \Ci(\FF)\pseries{\hbar}$. Then we have
  \begin{equation}\tag{extQME} \label{extendedQME}
    \left( dt \wedge \dd{}{t} - i \hbar \Delta_\mu \right) e^{i/\hbar \sigma} = 0.
  \end{equation}
  This equation is sometimes referred to as the "extended QME" (Mn\"{e}v) or as "homotopies of solutions of the QME" (Costello).
\end{definition}

\subsubsection{Quantum BV formalism}

\begin{definition}[Quantum BV Theory]
  A finite-dimensional \textbf{Quantum BV theory} is specified by the following set of data:
  \begin{enumerate}
    \item A $\ZZ$-graded manifold $\FF$.
    \item A $(-1)$-symplectic form $\Omega$ on $\FF$.
    \item A Berezinian $\mu \in BER(\FF)$, compatible with $\Omega$.
    \item A function $S \in \Ci(\FF)\pseries{\hbar}$ satisfying the \eqref{QME} (which is equivalent to $\delta_S := \{S, \cdot\}$ being a differential).
  \end{enumerate}
\end{definition}

\begin{rem}
  Note that the Hamiltionian vector field $X_S$ for the \textbf{full} BV quantum master function $S$ does \textbf{not} necessarily square to zero. However $\delta_S$ does and we know that $\delta_S \mod \hbar = X_{S^{(0)}} =: Q_0$ will square to zero which is nothing but the \eqref{ClassicalMasterEquation}. Thus we always have a classical BV theory.
\end{rem}

Given a Quantum BV theory we define the partition function to be
\begin{equation}
  \ZF_\LL := \int_{\LL \subset \FF} e^{\tfrac{i}{\hbar} S} \sqrt{\mu|_\LL},
\end{equation}
where $\LL$ is a Lagrangian submanifold of $\FF$. Now since $\Delta_\mu e^{\tfrac{i}{\hbar} S} = 0$ the value of $\ZF$ does not depend on a particular choice of $\LL$ inside the homotopy class $[\LL]$.\\

Now if $\FF = T^*[-1]F_{min}$ then $S$ is some lift of a classical action $S_{cl}$ on $F_{min}$. The zero section $\LL \equiv F_{min} \subset \FF$ is a valid Lagrangian submanifold but $\HH(S)$ is degenerate. Thus we deform to $\LL_0 \sim \LL$ such that $\HH(S)$ is no longer degenerate and compute the integral here. This sums up the concept of gauge-fixing.

\begin{rem}
  Consider $\OO \in \Ci(\FF)\pseries{\hbar}$. $\OO$ is an observable iff $\delta_S \OO = 0$. Then we can compute its expectation value to be
  \begin{equation}
    \langle \OO \rangle := \frac{1}{2} \int_{\LL \subset \FF} \OO e^{\tfrac{i}{\hbar} S} \sqrt{\mu|_\LL}.
  \end{equation}
  However a nontrivial fact is that this notion does not behave well under product because $\delta_S$ is not a derivation of the product of functions. The arising problems and the above definition are subject to current research.
\end{rem}

If we again take the BV-BRST data from above, thus $\FF := T^*[-1]F_{BRST}$ and $S_{BV} = p^* S_{cl} + d_{BRST}$ together with a function $\psi \in \Ci(F_{BRST})^{(-1)}$ we define
\begin{equation}
  \LL_\psi := \graph(d\psi)
\end{equation}
to obtain
\begin{align}
  S_{BV} |_{\LL_\psi} = S_{BV} \left( \Phi^a, \Phi^\dagger_a = \dell{}{\Phi^a} \psi \right) &= S_0 + \sum_a d^a_{BRST} \dell{}{\Phi^a}\psi \\
  &= S_0 + d_{BRST} \psi = S_{gf} \\
  \int_{\LL_\psi \subset T^*[-1]F_{BRST}} e^{ \tfrac{i}{\hbar} S_{BV}} \sqrt{\mu|_{\LL_\psi}} &= \int_{F_{BRST}} e^{\tfrac{i}{\hbar} S_{gf}} D\Phi
\end{align}

\begin{rem}
  To look at a $\psi$-dependent gauge-fixing in BRST we need the nonminimal extension $F_{Aux}$ which is the same as adding $T^*[-1]F_{Aux}$ to $T^*[-1]F_{min}$. In principle in BV we can directly look at Lagrangian submanifolds in $T^*[-1]F_{min}$. Thus we have no further need for this nonminimal extension.
\end{rem}

\subsection{BV Fibre Integrals}
Let $(\FF^{\prime}, \omega^{\prime})$ and $(\FF^{\prime\prime}, \omega^{\prime\prime})$ be two $(-1)$-symplectic manifolds and $\FF := \FF^\prime \times \FF^{\prime\prime}$ with $\omega := \omega^\prime \otimes 1 + 1 \otimes \omega^{\prime\prime} \in \Omega^\bullet(\FF^\prime) \otimes \Omega^\bullet(\FF^{\prime\prime})$.

\begin{definition}[BV Fibre Integral]
  IF $\LL \subset \FF^{\prime\prime}$ is a Lagrangian submanifold and if $\mu^{\prime\prime}$ is a Berezinian on $\FF^{\prime\prime}$ we have a map
  \begin{equation}
    P_*^{(\LL)} := \int_{\LL \subset \FF^{\prime\prime}} \cdot \ \sqrt{\mu^{\prime\prime}|_\LL} \colon \ \Ci(\FF) \lra \Ci(\FF^\prime)
  \end{equation}
  which is called a \textbf{BV fibre integral} or a \textbf{BV pushforward}.
\end{definition}

\begin{theo}[Stokes for BV fibre integrals]~
  \begin{enumerate}
    \item $P_*^{(\LL)}$ is a chain map relating the BV Laplacians $\Delta_\mu$ on $\FF$ and $\Delta_{\mu^\prime}$ on $\FF^\prime$, i.e.
    \begin{equation}
      \Delta_{\mu^\prime} \circ P_*^{(\LL)} = P_*^{(\LL)} \circ \Delta_\mu
    \end{equation}
    where $\mu = \mu^\prime \cdot \mu^{\prime\prime}$.

    \item If $\LL \sim \LL^\prime$ are Lagrangian submanifolds in $\FF^{\prime\prime}$ and $\rho \in \Ci(\FF)$ such that $\Delta_\mu \rho = 0$ then
    \begin{equation}
      P_*^{(\LL)} \rho - P_*^{(\LL^\prime)} \rho% = \Delta_{\mu^\prime}( – )
    \end{equation}
    is $\Delta_{\mu^\prime}$-exact in $\FF'$.
  \end{enumerate}
\end{theo}


\begin{definition}[Effective BV Actions]
  Let $\FF := \FF^\prime \times \FF^{\prime\prime}$ as above and $S \in \Ci(\FF)\pseries{\hbar}$ be a solution of the \eqref{QME} on $\FF$. Then we call $S^\prime \in \Ci(\FF^\prime)\pseries{\hbar}$ the \textbf{effective BV action} for $S$ induced on $\FF^\prime$ (given $\LL$) iff
  \begin{equation}
    e^{i/\hbar S^\prime} = P_*^{(\LL)} \left( e^{i/\hbar S} \right)
  \end{equation}
\end{definition}

\begin{corollary}
  Let $\FF, S$ and $S^\prime$ be as above, then:
  \begin{enumerate}
    \item If $S$ is a solution of the \eqref{QME} on $\FF$ then $S^\prime$ is a solution of the \eqref{QME} on $\FF^\prime$.

    \item If $S$ is a solution of the \eqref{QME} on $\FF$ and $\LL \sim \LL^\prime$ in $\FF^{\prime\prime}$ then the corresponding effective actions are equivalent (i.e. related by a canonical transformation) (w.r.t. $\Delta_{\mu^\prime}$).

    \item If $S \sim S^\prime$ are two solutions of the \eqref{QME} (equivalent by means of a canonical transformation), then their respective effective actions w.r.t. the same Lagrangian submanifold $\LL$ are also equivalent.
  \end{enumerate}
\end{corollary}

In short the BV fibre integral defines a map
\begin{equation}
  \Sol(QME(\FF))/\sim \quad \atob{P_*^{[\LL]}} \quad \Sol(QME(\FF^\prime))/\sim
\end{equation}
where $[\LL]$ is a homotopy class of Lagrangians in $\FF^{\prime\prime}$. We are going to use this fibre integral to interpret the renormalisation group flow/Wilson's effective action picture. Summing up we want a "tower" of theories
\begin{equation}
  (F,S) \lra \ ... \ \lra (F_\Lambda, S_\Lambda) \atob{P_*^{\Lambda \Lambda^\prime}} (F_{\Lambda^\prime}, S_{\Lambda^\prime}) \lra \ ... \ \lra (F_0, S_0)
\end{equation}
where on the left we have a local theory, in the middle we have finite-energy theories and on the right we have the zero modes. This will lead us to
\begin{equation}
  P_*^{\Lambda \Lambda^\prime} e^{i/\hbar S_\Lambda(\phi)} = \int_{[\Lambda, \Lambda^\prime]} D\widetilde{\phi} \ e^{i/\hbar S(\phi + \widetilde{\phi})} = e^{i/\hbar S_{\Lambda^\prime}(\phi)}
\end{equation}

\begin{rem}
  Note that $P_*^{[\LL]}$ is better thought of as an operator on Berezinians or even better on $1/2$-densities
  \begin{equation}
    \Ci(\MM) \atob{\sqrt{\mu}} Den^{1/2}(\MM) = \Gamma(\MM, BER(\MM)^{\otimes 1/2})
  \end{equation}
\end{rem}



\subsection{Field Theory}
Our goal in this chapter is to construct an $\infty$-dimensional generalisation of the BV construction. To this end consider the data of a fibre bundle $F \lra M$ where $F = \bigtimes_{i = - \infty}^{\infty} F_i$ such that $F_i \lra M$ is a vector bundle for all $i \neq 0$ and $F_0 \lra M$ is just a fibre bundle.\\
Now the space of BV fields is given by the sections $\FF = \Gamma(M, F)$ in the sense of classical field theory. Thus we know how to make sense of $J^\infty F$, $\Ci(J^\infty F)$, $C_{loc}^\infty(\FF)$ and so on.

\begin{rem}
  An important caveat in this setting is the definition of \textbf{duals}. We want to define $T^*[-1] \FF$ where $\FF = \Gamma(M, F)$. For simplicity we assume $F$ to be a vector bundle. The symbol $*$ in $F^*$ is used to denote the vector bundle $F^* := F^\vee \otimes Dens(M) \lra M$ where $F^\vee$ is the fibrewise dual. Thus define
  \begin{equation}
    \FF^* := \Gamma(M, F^*), \quad \quad T^*[-1]\FF := \FF \times \FF^*[-1]
  \end{equation}
  Alternatively one can use distributional sections and then consider the "strong dual" of the space of sections $\FF$. This ansatz is chosen in \emph{Perturbative Algebraic QFT}. With a little work one can extend the above ansatz to any fibre bundle $F \lra M$ by dualising $TF$ instead.
\end{rem}

The above remark lets us conclude the following: \textbf{Anti-fields} are "coordinates" in the cotangent fibre of $T^*[-1]\FF$ where $\FF = \Gamma(M, F)$. Thus $\phi^\dagger \in \Gamma(M, F^\vee \otimes Dens(M)) = \Gamma(M, F^*) = \FF^*$. Now there exists a nondegenerate (weak) pairing
\begin{align}
  T_\phi \FF \otimes T^*_\phi \FF &\lra \CC \\
  (v_\phi, \alpha_\phi) &\lmap \int_M \pair{v_\phi}{\alpha_\phi}
\end{align}
where $\pair{\cdot}{\cdot}$ is a $Dens(M)$-valued fibrewise pairing. This can be written as the local $2$-form
\begin{equation}
  \Omega = \int_M \pair{\delta \phi}{\delta \phi^\dagger}
\end{equation}
Then given a local Lagrangian field theory $(M, F, L)$ with $F \lra M$ a fibre bundle and $L \in \Omega_{loc}^{0,top}(\FF \times M)$ together with a set of local symmetries $\{V_a\}$ for now assumed to originate from a Lie algebra action
\begin{align}
  \rho \colon \gf &\lra \VF(\FF) \quad \text{for some Lie group} \ \ G\\
  \im(\rho) &\simeq \Gamma(M, \gf) \quad \quad M \times \gf \lra M
\end{align}

Note that this is a simplification, the generalisation works analogously to the previously discussed cases! In this setting the space of BV fields is the space of smooth sections of $T^*[-1](F \times \gf[1])$, i.e. $\phi \in \Gamma(M, F)$, $\phi^\dagger \in \Gamma(M, T^*F \otimes Dens(M))$, $c \in \Gamma(M, \gf[1])$, $c^\dagger \in \Gamma(M, \gf^*[-2] \otimes Dens(M))$. Then Ghost fields $c$ are functions on $M$ with values in the Lie algebra. For an even basis $\{t_a\}$ we have

\begin{equation}
 c = c^a (x) t_a \quad \quad |c^a| = 1 \quad \quad |t_a| = 0 \quad \quad |c| = 1
\end{equation}

Now we go an define the rest of the BV data

\begin{equation}
  \Omega_{BV} = \int_M \pair{\delta \phi}{\delta \phi^\dagger} = \int_M \pair{\delta \phi}{\delta \phi^\dagger} + \pair{\delta c^a}{\delta c^\dagger_a}.
\end{equation}

The BV action functional, if $S = \int_M L$, is given by

\begin{equation}
  S_{BV} = p^* S + \int_M \pair{\phi^\dagger}{Q_{CE} \phi}
\end{equation}

where $p \colon \FF_{BV} \lra \FF \times \Gamma(M, \gf[1])$ for $\FF_{BV} = T^*[-1](\FF \times \Gamma(M, \gf[1]))$. Further $Q_{CE}$ is a local (evolutionary) vector field on $\FF_{CE}$ such that

\begin{equation}
  Q_{CE}(\phi) = c^a V_a(\phi) \quad \quad Q_{CE} (c) = \frac{1}{2} [c,c].
\end{equation}

\subsubsection{Yang--Mills and Chern--Simons Theory}
\begin{example}[Yang--Mills theory]
  Let $(M,g)$ be a $4$-dimensional Riemannian manifold and let $\gf$ be a semisimple Lie algebra with an invariant (a.k.a.\ cyclic) inner product $\pair{\cdot}{\cdot}$. Here "invariant" means

  \begin{equation}
    \pair{X}{[Y,Z]} = \pair{[X,Y]}{Z} \quad \forall X,Y,Z \in \gf.
  \end{equation}

  Further assume that we have a trivial principal bundle $P \lra M$ such that the connections are identified with $\gf$-valued $1$-forms $\AC \simeq \Omega^1 (M, \gf) \simeq \Omega^1 (M) \otimes \gf$ thus $A \in \AC$ can be written as $A = A^a_i dx^i t_a$. The "gauge group" $G = \Ci(M, G)$ acts on $\AC$ by $(g, A) \lmap A^g := g^{-1} A g + g^{-1} dg$ and infinitesimally we have the Lie algebra action

  \begin{equation}
    \rho \colon (X,A) \lmap A + d_A X = A + dX + [A,X]
  \end{equation}

  where $X \in \gf \simeq \Ci(M, \gf)$ and $d_A X \in \Omega^1(M, \gf)$. Now the Yang-Mills action functional for $F_A := d A + \frac{1}{2} [A,A]$ is defined as
  \begin{equation}
    S_{YM_2} := \int_M \frac{1}{2} \pair{F_A}{* F_A}_\gf.
  \end{equation}
  Note that $\pair{\cdot}{\cdot}_\gf$ extends to a pairing on $\Omega^\bullet(M) \otimes \gf$ by denoting
  \begin{equation}
    \pair{\omega_1 \otimes E_1}{\omega_2 \otimes E_2} = \int \omega_1 \wedge \omega_2 \pair{E_1}{E_2}_\gf.
  \end{equation}

  \begin{lem}
    $S_{YM_2}$ is invariant under the Lie algebra action $\rho$.
  \begin{proof}
   Consider the vector field $d_A X \frac{\delta}{\delta A}$ representing $D_X(A) = d_A X$ on the algebra of functions. Now recall that $F_A = dA + \frac{1}{2} [A,A]$ such that
    \begin{align}
      D_X(F_A) &= d(d_A X) + [A, d_A X] = d[A,X] + [A,dX]
       [A,[A,X]]\\
       &= [dA,X] + \frac{1}{2}[[A,A], X] = [F_A, X]
    \end{align}
    But then we can compute
    \begin{align}
      \int_M D_X \left( \pair{F_A}{*F_A} \right) = \int_M \pair{[F_A, X]}{*F_A} + \pair{F_A}{ [*F_A, X]} = 0.
    \end{align}
    Thus the pairing is indeed invariant.
  \end{proof}
  \end{lem}

  Now we "promote" $X$ to a ghost field $C$ and look at $T^*[-1](\AC \times Lie(G)[1])$ with fields $((A,A^\dagger), (C, C^\dagger))$. Then define
  \begin{equation}
    \Omega_{BV} = \int_M \pair{\delta A}{\delta A^\dagger} + \pair{\delta C}{\delta C^\dagger}.
  \end{equation}
  Thus we arrive at the following BV action:

  \begin{equation}
    S^{BV}_{YM_2} = \int_M \frac{1}{2} \pair{F_A}{*F_A} + \pair{A^\dagger}{d_A C} + \frac{1}{2} \pair{C^\dagger}{[C,C]}.
  \end{equation}

  Assuming $\partial M = \emptyset$ we can compute $\imath_{Q_{BV}} \Omega_{BV} = \delta S_{BV}$ as well as
  \begin{align}
    Q_{BV} A &= d_A C,\\
    Q_{BV} C &= \frac{1}{2} [C,C], \\
    Q_{BV} A^\dagger &= d_A *F_A + [C,A^\dagger], \\
    Q_{BV} C^\dagger &= -d_A A^\dagger - [C,C^\dagger].
  \end{align}
  \begin{proof}
    \begin{align}
      \imath_Q \Omega = &\int_M \pair{Q_A}{\delta A^\dagger} + \pair{Q_{A^\dagger}}{\delta A} + \pair{Q_C}{\delta C^\dagger} + \pair{Q_{C^\dagger}}{\delta C} \\
      \delta S^{BV}_{YM_2} = &\int_M \pair{d_A * F_A}{\delta A} + \pair{\delta A^\dagger}{d_A C} - \pair{A^\dagger}{[\delta A, C]}\\
      &+ \pair{A^\dagger}{d_A \delta C} + \frac{1}{2} \pair{\delta C^\dagger}{[C,C]} - \pair{C^\dagger}{[C, \delta C]} \\
      = &\int_M \pair{d_A * F_A}{\delta A} + \pair{d_A C}{\delta A} + \pair{[A^\dagger,C]}{\delta A}\\
      &- \pair{d_A A^\dagger}{\delta C} + \frac{1}{2} \pair{[C,C]}{\delta C^\dagger} - \pair{[C^\dagger, C]}{\delta C} \\
    \end{align}
  \end{proof}

  \begin{rem}
    Note that for signs we use the total degree:
    \begin{align}
      \td(A) &= \deg(A) + \gr(A) = 1, \quad \quad \td(C) = \deg(C) + \gr(C) = 1\\
      \Lra \quad [A,C] &= [C,A], \quad \quad [C,C] \neq 0
    \end{align}
    Further note that $\td(\delta) = 1$ and $\td(d) = 1$ such that $\td(dA) = 2 = \td(\delta A)$.
  \end{rem}

  For clarity:
  \begin{align}
    \FF_{YM} = \Omega^1(M) \otimes \gf \times \Omega^0(M) \otimes \gf \times \Omega^3(M) \otimes \gf[-1] \times \Omega^{top}(M) \otimes \gf[-2]
  \end{align}
  where $T^*(\AC \Omega^\circ (M) \otimes \gf)$ such that $\pair{\phi}{\phi^\dagger} \in Dens(M)$. Now we can turn to a first order formulation of Yang-Mills theory:
  \begin{align}
    \FF_{YM_1} &= T^*[-1]\left( \Omega^1(M) \otimes \gf \times \Omega^2(M) \otimes \gf \times \Omega^0(M) \otimes \gf[-1] \right) \\
    \Omega_{YM_1} &= \int \pair{\delta A}{\delta A^\dagger} + \pair{\delta B}{\delta B^\dagger} + \pair{\delta C}{\delta C^\dagger}
  \end{align}
  Further we get the following action:
  \begin{equation}
    S_{YM_1} = \int \pair{B}{F_A} + \frac{1}{2} \pair{B}{*B} + \pair{A^\dagger}{d_A C} + \pair{B^\dagger}{[C,B]} + \pair{C^\dagger}{\frac{1}{2}[C,C]}
  \end{equation}
  And thus $F_A = *B$ and $d_A B = 0$ which yields $d_A *F_A = 0$. Now the Lie algebra action on $\Omega^2 (M) \otimes \gf$ is given by $B \lmap [X,B]$. Thus BV is promoted to $QB = [C,B]$.
  %TODO equations

  \begin{rem}
    As a graded vector space $\FF_{YM_1}$ is
    %TODO tikzcd diagram
  \end{rem}
\end{example}


\begin{example}[Chern Simons theory]
  Let $M$ be a $3$-dimensional oriented manifold with a trivial principal $G$-bundle $P \lra M$ over it. Then note $\gf = Lie(G)$ such that $\AC \simeq \Omega^1(M) \otimes \gf$. Now the action is given by
  \begin{equation}
    S_{CS}^0 = \int_M \frac{1}{2} \pair{A}{dA} + \frac{1}{6} \pair{A}{[A,A]}
  \end{equation}
  where $F_{CS} = \AC \simeq \Omega^1(M) \otimes \gf$. Now we get
  \begin{align}
    \delta L_{CS} &= \frac{1}{2} \pair{\delta A}{dA} + \frac{1}{2} \pair{A}{d \delta A} + \frac{1}{6} \pair{\delta A}{[A,A]} + \frac{1}{3} \pair{A}{[A,\delta A]} \\
    &= \pair{\delta A}{F_A} - d \left( \frac{1}{2} \pair{A}{\delta A} \right)
  \end{align}
  The latter term gives us a symplectic form on $\AC(\Sigma, \gf)$ while the former gives the equation of motion for CS theory. When $F_A = 0$ we have \textbf{flat connections}.

  \begin{ex}
    The usual Lie algebra action is $A \lmap A + d_A X$. Show that $S_{CS}$ is invariant under this action. Further compute the boundary form $\gamma = - \frac{1}{2} \pair{A}{\delta A}$.
  \end{ex}

  Now we can define
  \begin{equation}
    S_{CS} = S_{CS}^0 + \int_M \pair{A^\dagger}{d_A C} + \pair{C^\dagger}{\frac{1}{2}[C,C]}
  \end{equation}
  which, by defining $\AB := (C,A,A^\dagger) \in \Omega^\bullet (M) \otimes \gf[1- \bullet]$, can be rewritten as
  \begin{equation}
    S_{CS} = \int \frac{1}{2} \pair{\AB}{d\AB}^{top} + \frac{1}{6} \pair{\AB}{[\AB, \AB]}^{top}
  \end{equation}
\end{example}

Now the question is, how we can extend out previous quantum considerations into the BV setting for field theory. To this end, consider a gauge-fixing Lagrangian submanifold as the image of a "gauge-fixing operator" \cite{Costello}.

\subsection{Gauge fixing Lagrangian as gauge fixing operator}
\begin{definition}
  A \textbf{gauge-fixing operator} on a free BV theory, i.e. such that $S$ is a quadratic function, is given by
  \begin{equation}
    Q^{GF} \colon \FF \lra \FF
  \end{equation}
  such that the following conditions holds:
  \begin{enumerate}
    \item $|Q^{GF}| = -1$, $[Q^{GF}, Q^{GF}] = 0$ and it is self adjoint with respect to the pairing $\pair{\cdot}{\cdot}$.
    \item The commutator $[Q, Q^{GF}] =: D$ is a generalised Laplacian.
  \end{enumerate}
\end{definition}

\begin{rem}
  Scalar field theory can be formulated within the BV framework. Just take
  \begin{equation}
    \FF = T^*[-1] \Ci(M) \simeq \Ci(M) \oplus \Ci(M)[-1] \otimes Dens(M)
  \end{equation}
  with the following action:
  \begin{align}
    S_{BV} &= S_0 = \frac{1}{2} \int_M d\phi \wedge *_g d\phi \\
    Q_{BV} &\equiv d_\KK \quad \Lra \quad Q_{BV} \phi^\dagger = d *_g d\phi \equiv \Delta_g \phi \\
    Q_{BV} &\colon \FF^{(0)} \lra \FF^{(1)}
  \end{align}
  Thus the gauge-fixing operator is the identity seen as a map $Q^{GF} \colon \FF^{(1)} \lra \FF^{(0)}$ such that $[Q, Q^{GF}] = \Delta_g$. The Lagrangian submanifold $\FF^{(0)} \cong \Ci(M) \lra \FF$ is obtained as $\im(Q^{GF})$. Integrating on a BV path integral on this Lagrangian submanifold is the same as quantising scalar fields in the sense that we have previously seen.
\end{rem}

\subsubsection{Gauge fixing in $CS$ and $YM_1$}

\begin{theo}[Hodge decomposition]
  Let $(M,g)$ be a smooth Riemannian manifold. Then there exists an orthogonal decomposition
  \begin{equation}
    \Omega^k(M) = d\Omega^{k-1}(M) \oplus \HH^k(M) \oplus \delta \Omega^{k+1}(M)
  \end{equation}
  where $\HH^k(M) = \{\alpha \in \Omega^k(M) | \Delta_g \alpha = 0 \}$. If $M$ is compact, $\HH^k(M)$ is the $k$-th de Rham cohomology group of $M$, namely $\HH^k(M) = H^k_{dR}(M,d)$.
\end{theo}

\begin{rem}
  If $M$ is such that $H^k_{dR}(M,d) = 0$ for any $k > 0$ then
  \begin{align}
    \omega \ \ \text{closed} \quad &\Longleftrightarrow \quad \omega \ \ \text{exact} \\
    \omega \ \ \text{coclosed} \quad &\Longleftrightarrow \quad \omega \ \ \text{coexact}
  \end{align}
  Further we have
  \begin{align}
    \delta := * d * \colon \Omega^{k+1}(M) \lra \Omega^{k}(M) \\
    \Delta_g \equiv (d + \delta)^2 = d\delta + \delta d
  \end{align}
\end{rem}

\begin{prop}
  Assume $H^k_{dR}(M,d) = 0$ for any $k > 0$ and $M$ compact. Then $\ker(\delta) \subset \Omega^\bullet(M) \otimes \gf$ is a Lagrangian submanifold.
\begin{proof}
  This is rather a sketch of the proof: For a submanifold to be Lagrangian, it needs to be isotropic with an isotropic complement. Thus if $W = V \oplus V^\prime$ we have $V$ and $V^\prime$ isotropic. Now $\LL = \ker(\delta)$ is isotropic, i.e. $\Omega_{BV}$ vanishes on $\LL$. Thus for $\omega_2 = \delta \alpha_2$ we obtain
  \begin{align}
    &\pair{\omega_1}{\omega_2}|_{\ker(\delta)} = \pair{\omega_1}{\delta \alpha_2}|_{\ker(\delta)} = \pm \pair{\delta \omega_1}{\alpha_2}|_{\ker(\delta)} = 0
  \end{align}
  The same argument holds for $d$-exact and $d$-closed forms. Thus there exists a decomposition $\Omega^k(M) = d\text{-exact} \oplus \delta\text{-exact}$ into isotropic subspaces.
\end{proof}
\end{prop}

\begin{rem}
  When looking at manifolds with nontrivial cohomology, this becomes very complicated since
  $\Omega^\bullet(M) = d\Omega \oplus \delta \Omega \oplus \HH$
  where the first two spaces harbour the path integral and the third the "effective fields".
\end{rem}

A good gauge-fixing operator for $CS$-theory is precisely $\delta$ since $\delta^2 = 0$, $|\delta| = -1$, $\delta \colon \Omega^k \lra \Omega^{k-1}$ and $[Q_{BV}, \delta] = \Delta_g$ because $Q \equiv d$. Note that $Q_0 A = dC + [A,C]$. This sloppines comes from the fact that \textit{Costello} \cite{Costello} defines a "free" BV operator as given by the quadratic part of the BV action:
\begin{equation}
  S_{CS} = \int \frac{1}{2} \pair{A}{dA} + \pair{A^\dagger}{dC} + \frac{1}{6} \pair{A}{[A,A]} + \pair{A^\dagger}{[A,C]}
\end{equation}

\begin{rem}
  Chern Simons theory in principle does \textbf{not} require a specification of any metric in order to be defined and work. Thus it classifies as a so-called "Topological Theory" since it is independent of a metric. When working with gauge-fixing however, we need to fix a metric, even in this theory. Thus one needs to ask in how far the gauge-fixing procedure is independent of the choice of a metric. In his 1989 paper %TODO reference
  Witten has shown that one can remove "most" of the dependence on $g$. However it remains an open question if it is completely independent. Current belief is that it should be, although a full proof has yet to be developed. In $YM_1$ theory one can use the same gauge fixing using the same Riemannian structure $(M,g)$.
\end{rem}

\begin{notation}
  One typically calls $A \in \ker(\delta)$ the \textbf{Lorenz gauge} (namely thinking of $\ker(\delta)$ as an isotropic subspace of the fields) where one chooses $\delta A = 0$, thus $A \in \ker(\delta)$. This is typically written as $*d* A \lra \partial_\mu A^\mu = 0$, a form well-known from electrodynamics.
\end{notation}


\subsubsection{Quantum BV field theory}

Classical BV field theories makes perfect sense with minimal adaptations from the finite-dimensional case. This however turns out to be quite more complex for a quantum formulation. First note:
\begin{equation}
  \Delta_\mu f = - \frac{1}{2} \diver_\mu(X_f).
\end{equation}
However this \emph{requires} a Berezinian which in infinite dimensions is generally not defined! Now there exist several types of "regularisation attempts" to define $\Delta_\mu$. Some try to regularise the heat kernel \cite{Costello} while others probe triangulations/decompositions of manifolds into cells (a kind of finite-dimensional filter) \cite{Mnev, ...}. Again others start from local functionals and deform the product among functionals. As a "byproduct" they obtain "$\Delta$". This ansatz is called "Deformation Quantisation" and can be found in \cite{PAQFT}. In this subchapter we will only consider the first ansatz. Note that this is a very active field of research and while some approaches are more widespread, there is none that is considered the "right" approach.\\

Consider $\EE = \Gamma(M, E)$ where $E \lra M$ is a vector bundle, graded and endowed with a $(-1)$-symplectic pairing $\pair{\cdot}{\cdot}$ such that $\Omega_{BV}(V,W) = \pair{V}{W}$.

\begin{definition}
  Given any element $K \in \EE \otimes \EE$ we define a convolution
  \begin{equation}
    K* \colon \EE \lra \EE
  \end{equation}
  such that $\forall \psi \in \EE $ we have
  \begin{equation}
    K * \psi := (-1)^{|K|} ( 1 \otimes \pair{\cdot}{\cdot})(K \otimes \psi)
  \end{equation}
  Thus if $K = K_1 \otimes K_2$ we have $K * \psi = (-1)^{|K|} K_1 \pair{K_2}{\psi}$
\end{definition}

Now we denote $\widetilde{Q} = Q \otimes 1 + 1 \otimes Q$ on $\EE^{\otimes 2}$. This leads us to

\begin{lem}
  For $K$ as before we have $(\widetilde{Q} K) * \psi \equiv [Q, K *] \psi$.
\begin{proof}
  \begin{align}
    ((Q \otimes 1 + 1 \otimes Q ) K ) * \psi = \ \ &(QK_1 \otimes ) K_2 + (-1)^{|K_1|} K_1 \otimes Q K_2 ) * \psi \\
    = \ \ & (-1)^{|K|} QK_1 \pair{K_2}{\psi} + (-1)^{|K|} (-1)^{|K_1|} K_1 \pair{Q K_2}{\psi} \\
    = \ \ & Q \cdot (K*\psi) - (-1)^{|K|} K * Q \psi = ... \\
    = \ \ &(-1)^{|K|} QK_1 \pair{K_2}{\psi} + (-1)^{|K|} (-1)^{|K_1|} K_1 \pair{QK_2}{\psi}
  \end{align}
\end{proof}
\end{lem}

Now assume that we are given a gauge-fixing operator $Q^{GF}$ such that $D = [Q, Q^{GF}]$ is our generalised Laplacian. This is sufficient to make sense of the heat kernel of $e^{-LD}$. Let $K_L \in \EE \otimes \EE$ be defined as $K_L * \psi = \exp(-LD) \psi$.

\begin{definition}
  The (gauge-fixed) propagator is defined as
  \begin{equation}
    P(\epsilon, L) := \int_\epsilon^L (Q^{GF} \otimes 1) K_l \ dl
  \end{equation}
  such that $P(\epsilon,L) * \psi = Q^{GF} \int_\epsilon^l e^{-LD} \psi \ dl$. Thus it lands in $\in(Q^{GF})$.
\end{definition}

Now recall that given an element $K \in \EE^{\otimes 2}$ we can define a second-order operator as
\begin{equation}
  \partial_K := \dell{}{K_1} \otimes \dell{}{K_2}, \quad \quad \text{where} \ \ K = K_1 \otimes K_2
\end{equation}

\begin{definition}
  The scale-$L$ BV Laplacian is defined as
  \begin{equation}
    \Delta_L := - \partial_{K_L} \colon \Ci(\EE) \lra \Ci(\EE)
  \end{equation}
\end{definition}

\begin{definition}
  A functional $S \in \Ci(\EE)\pseries{\hbar}$ satisfies the scale-$L$ \eqref{QME} iff
  \begin{equation} \tag{$[QME]_L$} \label{QMEL}
    \Delta_L \exp\left( \tfrac{1}{\hbar} S \right) = 0 \quad \iff \quad \frac{1}{2} \{S,S\}_L + \hbar \Delta_L S = 0
  \end{equation}
  where
  \begin{equation}
    \{f,g\} := (-1)^{|f|} \Delta_L (fg) - (-1)^{|f|} \Delta_L f \cdot g - f \Delta_L g
  \end{equation}
  This means that we have a scale-$L$ BV algebra.
\end{definition}

Note that, in order to align with the respective parts in Costello \cite{Costello}, we did not choose a complex phase for the appearing exponentials. They are purely real.\\

It is often convenient to split $S$ into a \emph{free} (quadratic) part $S_0$ and an \emph{interaction} (cubic) part $I$ such that $S = S_0 + I$. Now choosing $Q_0 := \{S_0, \cdot \}$ we have a \emph{classical free BV operator}. Thus we can reformulate the definition of the \eqref{QME} for interaction terms. $I$ satisfies \eqref{QMEL} iff
\begin{equation}\tag{$[I-QME]_L$} \label{IQMEL}
  (Q_0 + \Delta_L) \exp(1/\hbar I) = 0 \quad \iff \quad Q_0 I + \frac{1}{2} \{I,I\}_L + \hbar \Delta_L I = 0
\end{equation}

\begin{definition}[Pre-Theories]
  A \textbf{Pre-Theory} is a collection of effective interactions $\{I[L]\}$ such that the following conditions hold:
  \begin{enumerate}
    \item Each $I \in \Ci(\EE)\pseries{\hbar}$ is of degree $0$ and at least cubic modulo $\hbar$.
    \item The $RGE$ $I[L] = W(P(\epsilon, L), I[\epsilon])$ is satisfied, where we defined \\$W(P(\epsilon, L), I[\epsilon]) = \hbar \log \left( \exp(\hbar \partial_{P(\epsilon,L)}) e^{I/\hbar} \right)$ \ref{lemma:WPI} (formally analogous to the scalar field theory case).
    \item Each $I_{i,k}[L]$ has a small-$L$ asymptotic expansion in terms of local functionals, where $(i,k)$ indicate the homogeneous part of degree $k$ of the $\hbar^i$ coefficient.
  \end{enumerate}
  Further, a \textbf{Theory} is a Pre-Theory such that $I[L]$ satisfies \eqref{IQMEL} for all $L$. The space of Pre-Theories forms a presheaf.
\end{definition}

\begin{prop}
  A functional $I \in \Ci(\EE)\pseries{\hbar}$ satisfies $[QME]_\epsilon$ iff it satisfies $[QME]_L$. Note that this is the same as the BV theorem for fibre integrals.
\begin{proof}
  Note that $P(\epsilon,L)$ is the kernel of $\int_\epsilon^L Q^{GF} e^{-lD} \ dl$. Now this lets us denote
  \begin{equation}
    P * \psi = \int_\epsilon^L (Q^{GF} \otimes 1) K_l \ dl * \psi ) \int_\epsilon^L Q^{GF} e^{-lD} \psi
  \end{equation}
  Thus we can compute
  \begin{align}
    \left[Q, \int_\epsilon^L Q^{GF} e^{-lD} \ dl \right] &= \int_\epsilon^L [Q,Q^{GF}] e^{-lD} \ dl \\
    &= \int_\epsilon^L D e^{-lD} \ dl = e^{-\epsilon D} - e^{-l D} \\
    \Lra \quad (K_\epsilon - K_l) * \psi &= \left( e^{-\epsilon D} - e^{-l D} \right) \psi
  \end{align}
  Using the last lemma we obtain
  \begin{align}
    (\widetilde{Q} P(\epsilon,L)) * \psi &= [Q, P *] \psi \\
    \Lra \quad \widetilde{Q} P &= K_\epsilon - K_L
  \end{align}
  Now passing to second-order operators we get
  \begin{equation}
    [Q, \partial_{P(\epsilon,L)}] = \partial_{K_\epsilon} - \partial_{K_L} = \Delta_L - \Delta_\epsilon
  \end{equation}
  Thus $\partial_{P(\epsilon,L)}$ can be seen as a chain homotopy between $\Delta_L$ and $\Delta_\epsilon$. Thus
  \begin{align}
    &(Q + \hbar \Delta_L) \exp(W(P(\epsilon,L)I)/\hbar)\\
    = \ \ &(Q + \hbar \Delta_L) (\exp(\hbar \partial_P) \exp(I/\hbar))\\
    = \ \ &\exp(\hbar \partial_P) (Q + \hbar \Delta_\epsilon) \exp(I/\hbar)
  \end{align}
\end{proof}

\end{prop}

This last result of the course poses as an entry point for even more sophisticated discussions in \textit{Costello} \cite{Costello} and other sources.

~\\
