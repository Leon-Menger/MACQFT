% Document type and font size
\documentclass[12pt, twoside]{article}
    \setlength{\parskip}{0pt}

\usepackage[left=2cm,right=2cm,top=2.5cm,bottom=2.5cm]{geometry}


% The most important definition overall
\let\oldphi\varphi \let\varphi\phi \let\phi\oldphi
\let\oldepsilon\varepsilon \let\varepsilon\epsilon \let\epsilon\oldepsilon


% List of used packages (insanely many if you ask me)
\usepackage{amsmath}
\usepackage{amsfonts}
\usepackage[utf8]{inputenc}
\usepackage{graphicx}
\usepackage[T1]{fontenc}
\usepackage{mathtools}
\usepackage[english]{babel}
\usepackage{graphicx}
\usepackage{fancyhdr}
\usepackage{musicography}
\setlength{\headheight}{15pt}
\usepackage{float}
\usepackage{afterpage}
\usepackage{placeins}
\usepackage{esvect}
\usepackage{multicol}
\usepackage{titlesec}
\usepackage{setspace}
\usepackage[font=small,labelfont=bf]{caption}
\usepackage{url}
\usepackage{musicography}
\usepackage{tikz-cd}
\usepackage{dsfont}
\usepackage{stmaryrd}
\usepackage{mdframed}
\usepackage[normalem]{ulem}
\usepackage{amssymb}

\usepackage{amsthm}
\DeclareMathAlphabet{\mymathbb}{U}{BOONDOX-ds}{m}{n}



% Packages required to render template
\usepackage{xstring}
\usepackage{subfigure}
\usepackage{multirow}
\usepackage{hyperref}



% General settings for section number display
\setcounter{secnumdepth}{4}
\setcounter{tocdepth}{3}
\setcounter{section}{-1}
\setlength{\parindent}{0em}



% Specifies the picture path
\graphicspath{
	{../pics/}
}



% Allow for more dead cycles (too lazy to fix)
% Error can occur if ~30+ pictures are displayed
\maxdeadcycles=200



% Command for quotes (often problematic when writing with ngerman)
\newcommand{\q}[1]{\glqq #1\grqq{}}



% Command that sums up the insertion of an image
\newcommand{\img}[4][width=0.9\textwidth]{
	\begin{figure}[ht]
	\centering
	\includegraphics[#1]{#2}
	\caption[#4]{#3}
	\label{fig:#2}
	\end{figure}
}



% Commands to make citation and referencing look better
\newcommand{\Abb}[1]{\textbf{figure \ref{fig:#1}}}
\newcommand{\source}[1]{[\ref{#1}]}
\newcommand{\dref}[1]{\textit{definition \ref{#1}}}
\newcommand{\ssecref}[1]{\textit{subsection \ref{subsec:#1}}}
\newcommand{\secref}[1]{\textit{section \ref{sec:#1}}}
\newcommand{\remref}[1]{\textit{remark \ref{rem:#1}}}
\newcommand{\exref}[1]{\textit{example \ref{ex:#1}}}



% Commands to abbreviate certain script specific notation
\newcommand{\ABS}{\\ \\ \\}
\newcommand{\ra}{\rightarrow}
\newcommand{\lra}{\longrightarrow}
\newcommand{\Lra}{\Longrightarrow}
\newcommand{\la}{\leftarrow}
\newcommand{\lla}{\longleftarrow}
\newcommand{\lmap}{\longmapsto}
\newcommand{\map}{\mapsto}
\newcommand{\RR}{\ensuremath{\mathbb{R}}}
\newcommand{\QQ}{\ensuremath{\mathbb{Q}}}
\newcommand{\JJ}{\ensuremath{\mathfrak{J}}}
\newcommand{\ZF}{\ensuremath{\mathcal{Z}}}
\newcommand{\NC}{\ensuremath{\mathcal{N}}}
\newcommand{\AC}{\ensuremath{\mathcal{A}}}
\newcommand{\AB}{\ensuremath{\mathbb{A}}}
\newcommand{\TT}{\ensuremath{\mathfrak{T}}}
\newcommand{\PP}{\ensuremath{\mathbb{P}}}
\newcommand{\OO}{\ensuremath{\mathcal{O}}}
\newcommand{\BB}{\ensuremath{\mathcal{B}}}
\newcommand{\MM}{\ensuremath{\mathcal{M}}}
\newcommand{\RC}{\ensuremath{\mathcal{R}}}
\newcommand{\LL}{\ensuremath{\mathcal{L}}}
\newcommand{\NN}{\ensuremath{\mathbb{N}}}
\newcommand{\ABF}{\ensuremath{\mathbb{A}}}
\newcommand{\ZZ}{\ensuremath{\mathbb{Z}}}
\newcommand{\CC}{\ensuremath{\mathbb{C}}}
\newcommand{\Ck}{\ensuremath{\mathcal{C}^{k}}}
\newcommand{\Ci}{\ensuremath{\mathcal{C}^{\infty}}}
\newcommand{\RP}{\ensuremath{\mathbb{R}\mathbb{P}^n}}
\newcommand{\dd}[2]{\ensuremath{\frac{d #1}{d #2}}}
\newcommand{\dell}[2]{\ensuremath{\frac{\partial #1}{\partial #2}}}
\newcommand{\TM}{\ensuremath{T^*M}}
\newcommand{\VF}{\ensuremath{\mathfrak{X}}}
\newcommand{\TR}{\ensuremath{T^*\mathbb{R}}}
\newcommand{\HH}{\ensuremath{{\mathcal{H}}}}
\newcommand{\LE}{\ensuremath{{\mathcal{L}_e}}}
\newcommand{\DD}{\ensuremath{\mathfrak{D}}}
\newcommand{\FF}{\ensuremath{{\mathcal{F}}}}
\newcommand{\EE}{\ensuremath{{\mathcal{E}}}}
\newcommand{\KK}{\ensuremath{{\mathcal{K}}}}
\newcommand{\KT}{\ensuremath{{\mathcal{KT}}}}
\newcommand{\GG}{\ensuremath{{\mathcal{G}}}}
\newcommand{\fg}{\ensuremath{\mathfrak{g}}}
\newcommand{\MT}{\ensuremath{\widetilde{M}}}
\newcommand{\Tp}{\ensuremath{T_p M}}
\newcommand{\Tps}{\ensuremath{T^*_p M}}
\newcommand{\TMT}{T^* \widetilde{M}}
\newcommand{\TNT}{T^* \widetilde{N}}
\newcommand{\inp}{\mathbin{\raisebox{\depth}{\scalebox{1}[-1]{$\lnot$}}}}
\newcommand{\lie}[1]{\mathcal{L}_{#1}}
\newcommand{\OT}{\widetilde{\omega}}
\newcommand{\kfs}{\textit{k-forms }}
\newcommand{\kf}{\textit{k-form }}
\newcommand{\tens}[1]{%
  \mathbin{\mathop{\otimes}\displaylimits_{#1}}%
}
\newcommand{\TMN}{\ensuremath{\mathbb{T}^m_n M}}
\newcommand{\gf}{\ensuremath{\mathfrak{g}}}
\newcommand{\tf}{\ensuremath{\mathfrak{t}}}
\newcommand{\pair}[2]{\ensuremath{\left\langle #1 , #2 \right\rangle}}
\newcommand{\lrangle}[1]{\ensuremath{\left\langle #1 \right\rangle}}
\newcommand{\atob}[1]{\ensuremath{\overset{#1}{\lra}}}
\newcommand{\limit}[1]{\ensuremath{\underset{#1}{lim}}}
\newcommand{\nequiv}{\ensuremath{\not\equiv}}
\newcommand{\pseries}[1]{\ensuremath{\llbracket #1 \rrbracket}}

\DeclareMathOperator\arctanh{arctanh}
\DeclareMathOperator\End{End}
\DeclareMathOperator\Hom{Hom}
\DeclareMathOperator\Aut{Aut}
\DeclareMathOperator\Der{Der}
\DeclareMathOperator\id{id}
\DeclareMathOperator\degr{deg}
\DeclareMathOperator\im{im}
\DeclareMathOperator\vol{Vol}
\DeclareMathOperator\sgn{sgn}
\DeclareMathOperator\Tr{Tr}
\DeclareMathOperator\Sym{Sym}
\DeclareMathOperator\sing{sing}
\DeclareMathOperator\reg{reg}
\DeclareMathOperator\sdet{sdet}
\DeclareMathOperator\tr{tr}
\DeclareMathOperator\str{str}
\DeclareMathOperator\diver{div}
\DeclareMathOperator\gr{gr}
\DeclareMathOperator\pr{pr}
\DeclareMathOperator\td{td}
\DeclareMathOperator\graph{graph}
\DeclareMathOperator\Span{span}
\DeclareMathOperator\Crit{Crit}
\DeclareMathOperator\Sol{Sol}
\DeclareMathOperator\gra{gr}
\DeclareMathOperator\Mett{Met}
\DeclareMathOperator\Diff{Diff}

\numberwithin{equation}{section} % restart equation numbering with each section

\newcommand{\iam}[1]{\vspace{.1cm}{\fbox{\textbf{#1}}}\vspace{.1cm}}

% Definition of common mathematical environments
\raggedbottom
\newtheoremstyle{note}%hnamei
{1.2em}              %   Space above
{1.2em}              %   Space below
{}                     %   Body font
{}                     %   Indent amount
{\scshape\bfseries}    %   Theorem head font
{ }                    %   Punctuation after theorem head
{.5em}                 %   Space after theorem head
{}                     %   Theorem head spec (can be left empty, meaning `normal')
\theoremstyle{note}
\newtheorem{theo}{Theorem}[section]
\newtheorem{definition}[theo]{Definition}
\newtheorem{lem}[theo]{Lemma}
\newtheorem{notation}[theo]{Notation}
\newtheorem{corollary}[theo]{Corollary}
\newtheorem{prop}[theo]{Proposition}
\newtheorem{ex}[theo]{Exercise}
\newtheorem{example}[theo]{Example}
\newtheorem{sketch}[theo]{Sketch}
\newtheorem{fact}[theo]{Fact}
\newtheorem{rem}[theo]{\textit{Remark}}

\newcounter{PRINCIPLE}
\newtheorem{principle}[PRINCIPLE]{Principle}



% Fancy pagestyle for header/footer
\pagestyle{fancy}



% Some changes in header and footer formatting
\renewcommand{\sectionmark}[1]{\gdef\currsection{#1}}
\renewcommand{\subsectionmark}[1]{\markright{\currsection \ \ $\cdot$ \ \  #1}}
\fancyhead{}
\renewcommand{\headrulewidth}{0.4pt}
\fancyhead[LE,RO]{\thepage}
\fancyhead[RE,LO]{\nouppercase{\rightmark}}
\renewcommand{\headrulewidth}{0.1pt}
\renewcommand{\footrulewidth}{0.1pt}
\renewcommand{\headrulewidth}{0.1pt}



% Stretch artificially changes the space between letters
% Use at own discretion (everything above 1.2 looks horrible)
\setstretch{1.1}



% Finally start the document
